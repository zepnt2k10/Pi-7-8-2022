 \thispagestyle{cackithitoannone}
\pagestyle{cackithitoan}
\everymath{\color{cackithi}}
\graphicspath{{../cackithi/pic/}}
%\blfootnote{{\color[named]{cackithi}$^1$Trường THPT chuyên Khoa học Tự nhiên, Đại học KHTN, Đại học Quốc Gia Hà Nội.}}
\begingroup
\AddToShipoutPicture*{\put(0,616){\includegraphics[width=19.3cm]{../bannercackithi}}} 
\AddToShipoutPicture*{\put(120,525){\includegraphics[scale=1]{../tieude.pdf}}} 
\centering
\endgroup
\vspace*{180pt}

%\begin{multicols}{2}
%	\textbf{\color{cackithi}Vài nét về kỳ thi}
%	\vskip 0.05cm
%	Đây là kỳ thi được tổ chức lần đầu tiên vào năm $2000$, do bộ Giáo dục và Đào tạo Pháp và hiệp hội Animath khởi xướng, nhằm mục đích phát triển sự tò mò, khả năng tìm tòi cũng như tư duy phản biện của học sinh thông qua việc xử lý hoặc cá nhân hoặc theo nhóm một số bài toán cụ thể. Bên cạnh đó, kỳ thi còn nhằm mục đích phát triển và nâng cao văn hóa khoa học và công nghệ, nhằm mục đích kích thích sự sáng tạo và chủ động của học sinh, khơi dậy trong học sinh niềm yêu thích với bộ môn Toán. Đối tượng mà kỳ thi hướng tới là những học sinh khối lớp $11$ theo chương trình phổ thông Pháp, ở cả trong và ngoài nước. 
%	\vskip 0.05cm
%	Học sinh sẽ trải qua hai bài thi độc lập, mỗi bài thi kéo dài $120$ phút. Tùy theo chương trình mà học sinh theo học, chương trình chuyên biệt hay chương trình phổ quát, học sinh chọn hai trong ba bài tập được đưa ra. Bài thi thứ nhất là phần thi cá nhân bao gồm những bài tập thuộc cấp quốc gia. Bài thi số hai gồm các bài tập thuộc cấp tỉnh, điểm đặc biệt của bài thi này là học sinh có thể chọn hoặc thi đơn hoặc thi theo nhóm từ $2$ tới $3$ thí sinh thậm chí các thành viên trong nhóm có thể tới từ những lớp khác nhau trong cùng một khối, đây là một trong những điểm đổi mới của kỳ thi Olympic Toán so với những kỳ thi Toán khác nhằm mục đích khuyến khích và phát huy khả năng làm việc nhóm của học sinh.
%	\vskip 0.05cm 
%	Hội đồng chấm thi được chỉ định và chia thành hai cấp: quốc gia và cấp tỉnh. Trong đó hội đồng cấp tỉnh một mặt chấm bài thi cấp tỉnh của các thí sinh ở tỉnh đó nhằm chọn ra những bài làm xuất sắc nhất để trao huy chương, bên cạnh đó còn chọn lựa từ những bài thi cấp quốc gia của tỉnh mình những bài làm xuất sắc nhất để gửi lên hội đồng cấp quốc gia. Qua đó, hội đồng cấp quốc gia xem xét và chọn ra trong những đề cử từ các tỉnh khác nhau những bài làm xuất sắc nhất để trao huy chương. 
%	\vskip 0.05cm
%	Kỳ thi Olympic Toán lần thứ $22$ tại Pháp, diễn ra vào ngày $14/03/2022$ đã thu hút sự tham gia của $22000$ học sinh khối $11$ từ các trường trung học trên các tỉnh thành của khắp nước Pháp, cũng như từ những trường trung học Pháp ở nước ngoài. Sau khi thảo luận thống nhất vào ngày $25/05/2022$, hội đồng cấp quốc gia đã chọn ra $50$ bài làm xuất sắc nhất để trao giải nhất, nhì, ba cùng các huy chương và giấy chứng nhận của Bộ Giáo dục và Đào tạo Pháp. Dưới sự tài trợ của hiệp hội Animath, Viện nghiên cứu Tin học và Tự động  INRIA, Trường Bách Khoa Paris, các tập đoàn Texas instrument, Casio, Credit mutuel, cũng như từ các trường trung học có thí sinh tham gia, những học sinh đạt giải cấp quốc gia cùng gia đình và giáo viên Toán đã được mời tham dự lễ tổng kết trao giải đã diễn ra vào ngày $08/06/2022$ tại Viện nghiên cứu về thế giới Ả rập, Paris. Qua đó, học sinh có cơ hội để tìm hiểu về lịch sử của kỳ thi Olympic Toán, gặp gỡ những nhà toán học hàng đầu nước Pháp, nghe báo cáo của các giáo sư và nghiên cứu sinh đang nghiên cứu về Toán \ldots 
%	\vskip 0.05cm
%	\textbf{\color{cackithi}Đề thi Olympic Toán quốc gia năm $\pmb{2022}$ dành cho khu vực Pháp -- Châu Âu -- Châu Phi -- Đông Ấn} 
%	\vskip 0.05cm
%	Để đáp ứng các tiêu chí đã nêu của kỳ thi Olympic Toán, các đề thi được chọn lựa kỹ càng sao cho mỗi học sinh đều cảm thấy hứng thú trong việc tìm kiếm lời giải cho những bài toán mở cũng như góp phần vào việc phát huy khả năng tìm tòi, sáng tạo của từng học sinh. Vì việc lệch múi giờ giữa các địa điểm thi nên ban ra đề thi đã soạn ra ba đề thi khác nhau, tương ứng với các khu vực địa lý khác nhau, bao gồm khu vực Pháp -- Châu Âu -- Châu Phi -- Đông Ấn, khu vực  Châu Mỹ -- Antilles -- Guyana và khu vực  Châu Á -- Thái Bình Dương. Trong khuôn khổ của bài viết, các tác giả chỉ giới thiệu tới bạn đọc đề thi dành cho khu vực Pháp -- Châu Âu -- Châu Phi -- Đông Ấn, độc giả có nhu cầu tìm hiểu thêm có thể tham khảo tại đường dẫn dưới đây:
%	\vskip 0.05cm
%	https://www.freemaths.fr/annales-olympiad\\es-mathematiques-premieres-scientifiques-s\\/mathematiques-olympiades-epreuve-nation\\ale-enonce-2022.pdf
%	\vskip 0.05cm
%	\textbf{\color{cackithi}Bài $\pmb{1}$ (Chung cho tất cả thí sinh)}
%	\vskip 0.05cm
%	\textbf{\color{cackithi}Dán nhãn duyên dáng của một hình}
%	\vskip 0.05cm
%	Xét một tập hợp hữu hạn các điểm. Ta nối một số điểm trong số những điểm đã cho bởi những đoạn thẳng. Tập hợp được tạo ra theo cách đó được gọi là \textit{hình}.
%	\vskip 0.05cm 
%	Ta thực hiện việc \textit{dán nhãn} của một hình gồm $n$ đoạn thẳng bằng cách gắn mỗi đỉnh của hình đó với một số tự nhiên đôi một khác nhau trong khoảng từ $0$ tới $n$.
%	\vskip 0.05cm 
%	Mỗi đoạn thẳng được gán với giá trị tuyệt đối của hiệu giữa hai số tự nhiên được gắn cho hai đầu mút của đoạn thẳng đó. Giá trị tuyệt đối thu được là một số tự nhiên, được gọi là \textit{trọng số} của đoạn thẳng. 
%	\vskip 0.05cm
%	Ta nói rằng sự dán nhãn của một hình là \textit{duyên dáng} nếu $n$ trọng số nhận được trên các đoạn thẳng là những số tự nhiên từ $1$ tới~$n$.
%	\vskip 0.05cm 
%	Dưới đây là một ví dụ về sự dán nhãn duyên dáng cho một hình gồm $6$ điểm và $7$ đoạn thẳng.	
%	\begin{figure}[H]
%		\vspace*{-15pt}
%		\centering
%		\captionsetup{labelformat= empty, justification=centering}
%		\begin{tikzpicture}[scale=0.8]
%				\draw [cackithi,line width=0.8pt] (0.,2.)-- (4.,2.);
%				\draw [cackithi,line width=0.8pt] (4.,2.)-- (4.,0.);
%				\draw [cackithi,line width=0.8pt] (4.,0.)-- (0.,0.);
%				\draw [cackithi,line width=0.8pt] (0.,0.)-- (0.,2.);
%				\draw [cackithi,line width=0.8pt] (2.,2.)-- (2.,0.);
%	
%				\draw [fill=cackithi] (0.,2.) circle (2.5pt);
%				\draw[color=cackithi] (-0.22,2.45) node {$0$};
%				\draw [fill=cackithi] (4.,2.) circle (2.5pt);
%				\draw[color=cackithi] (4.24,2.45) node {$2$};
%				\draw [fill=cackithi] (4.,0.) circle (2.5pt);
%				\draw[color=cackithi] (4.22,-0.4) node {$6$};
%				\draw [fill=cackithi] (0.,0.) circle (2.5pt);
%				\draw[color=cackithi] (-0.2,-0.4) node {$7$};
%				\draw [fill=cackithi] (2.,2.) circle (2.5pt);
%				\draw[color=cackithi] (2.,2.45) node {$3$};
%				\draw [fill=cackithi] (2.,0.) circle (2.5pt);
%				\draw[color=cackithi] (1.98,-0.4) node {$1$};
%		\end{tikzpicture}
%	
%		\vspace*{-5pt}
%		\caption{\small\textit{\color{cackithi}Hình được dán nhãn.}}
%		\vspace*{-10pt}
%	\end{figure}
%	\begin{figure}[H]
%		\vspace*{-10pt}
%		\centering
%		\captionsetup{labelformat= empty, justification=centering}
%		\begin{tikzpicture}[scale=0.8]
%			\draw [cackithi,line width=0.8pt] (0.,2.)-- (4.,2.);
%			\draw [cackithi,line width=0.8pt] (4.,2.)-- (4.,0.);
%			\draw [cackithi,line width=0.8pt] (4.,0.)-- (0.,0.);
%			\draw [cackithi,line width=0.8pt] (0.,0.)-- (0.,2.);
%			\draw [cackithi,line width=0.8pt] (2.,2.)-- (2.,0.);
%			
%			\draw [fill=cackithi] (0.,2.) circle (2.5pt);
%			\draw[color=cackithi] (-0.22,2.45) node {$0$};
%			\draw(1,2.5) node[squarednode]{$3$};
%			\draw [fill=cackithi] (4.,2.) circle (2.5pt);
%			\draw[color=cackithi] (4.24,2.45) node {$2$};
%			\draw(3,2.5) node[squarednode]{$1$};
%			\draw [fill=cackithi] (4.,0.) circle (2.5pt);
%			\draw[color=cackithi] (4.22,-0.4) node {$6$};
%			\draw(3,-0.5) node[squarednode]{$5$};
%			\draw(1,-0.5) node[squarednode]{$6$};
%			\draw(-0.5,1) node[squarednode]{$7$};
%			\draw(1.5,1) node[squarednode]{$2$};
%			\draw(4.5,1) node[squarednode]{$4$};
%			\draw [fill=cackithi] (0.,0.) circle (2.5pt);
%			\draw[color=cackithi] (-0.2,-0.4) node {$7$};
%			\draw [fill=cackithi] (2.,2.) circle (2.5pt);
%			\draw[color=cackithi] (2.,2.45) node {$3$};
%			\draw [fill=cackithi] (2.,0.) circle (2.5pt);
%			\draw[color=cackithi] (1.98,-0.4) node {$1$};
%		\end{tikzpicture}
%	
%		\vspace*{-5pt}
%		\caption{\small\textit{\color{cackithi}Hình được dán nhãn và trọng số.}}
%		\vspace*{-10pt}
%	\end{figure}
%	\textbf{\color{cackithi}A. Một vài ví dụ}
%	\vskip 0.05cm
%	$1.$ Trong các hình dưới đây, hình nào cho ta một dán nhãn duyên dáng ?
%	\begin{figure}[H]
%		\vspace*{-15pt}
%		\centering
%		\captionsetup{labelformat= empty, justification=centering}
%		\begin{tikzpicture}[scale=0.8]
%			\draw [cackithi,line width=0.8pt] (0.,2.)-- (4.,2.);
%			\draw [cackithi,line width=0.8pt] (4.,2.)-- (4.,0.);
%			\draw [cackithi,line width=0.8pt] (4.,0.)-- (0.,0.);
%			\draw [cackithi,line width=0.8pt] (0.,0.)-- (0.,2.);
%			\draw [cackithi,line width=0.8pt] (2.,2.)-- (2.,0.);
%			
%			\draw [fill=cackithi] (0.,2.) circle (2.5pt);
%			\draw[color=cackithi] (-0.22,2.45) node {$0$};
%			\draw [fill=cackithi] (4.,2.) circle (2.5pt);
%			\draw[color=cackithi] (4.24,2.45) node {$3$};
%			\draw [fill=cackithi] (4.,0.) circle (2.5pt);
%			\draw[color=cackithi] (4.22,-0.4) node {$5$};
%			\draw [fill=cackithi] (0.,0.) circle (2.5pt);
%			\draw[color=cackithi] (-0.2,-0.4) node {$7$};
%			\draw [fill=cackithi] (2.,2.) circle (2.5pt);
%			\draw[color=cackithi] (2.,2.45) node {$6$};
%			\draw [fill=cackithi] (2.,0.) circle (2.5pt);
%			\draw[color=cackithi] (1.98,-0.4) node {$1$};
%		\end{tikzpicture}
%		\vspace*{-10pt}
%	\end{figure}
%	\begin{figure}[H]
%		\vspace*{-10pt}
%		\centering
%		\captionsetup{labelformat= empty, justification=centering}
%		\begin{tikzpicture}[scale=0.8]
%			\draw [cackithi, line width=0.8pt] (0.,2.)-- (4.,2.);
%			\draw [cackithi,line width=0.8pt] (4.,2.)-- (4.,0.);
%			\draw [cackithi,line width=0.8pt] (4.,0.)-- (0.,0.);
%			\draw [cackithi,line width=0.8pt] (0.,0.)-- (0.,2.);
%			\draw [cackithi,line width=0.8pt] (2.,2.)-- (2.,0.);
%			
%			\draw [fill=cackithi] (0.,2.) circle (2.5pt);
%			\draw[color=cackithi] (-0.22,2.45) node {$0$};
%			\draw [fill=cackithi] (4.,2.) circle (2.5pt);
%			\draw[color=cackithi] (4.24,2.45) node {$2$};
%			\draw [fill=cackithi] (4.,0.) circle (2.5pt);
%			\draw[color=cackithi] (4.22,-0.4) node {$4$};
%			\draw [fill=cackithi] (0.,0.) circle (2.5pt);
%			\draw[color=cackithi] (-0.2,-0.4) node {$7$};
%			\draw [fill=cackithi] (2.,2.) circle (2.5pt);
%			\draw[color=cackithi] (2.,2.45) node {$7$};
%			\draw [fill=cackithi] (2.,0.) circle (2.5pt);
%			\draw[color=cackithi] (1.98,-0.4) node {$2$};
%		\end{tikzpicture}
%		\vspace*{-15pt}
%	\end{figure}
%	$2.$ Bổ sung hình sau để được một dán nhãn duyên dáng.
%	\begin{figure}[H]
%%		\vspace*{-15pt}
%		\centering
%		\captionsetup{labelformat= empty, justification=centering}
%		\begin{tikzpicture}[scale=0.85]
%				\draw [cackithi,line width=0.75pt] (-1.6180339887498947,2.9021130325903073)-- (0.,2.38);
%				\draw [cackithi,line width=0.8pt] (0.,2.38)-- (-1.,1.);
%				\draw [cackithi,line width=0.8pt] (0.,2.38)-- (1.,1.);
%				\draw [cackithi,line width=0.8pt] (0.,2.38)-- (1.618033988749895,2.9021130325903064);
%				\draw [cackithi,line width=0.8pt] (0.,4.077683537175253)-- (0.,2.38);
%				\draw [cackithi,line width=0.8pt] (0.,4.077683537175253)-- (-1.6180339887498947,2.9021130325903073);
%				\draw [cackithi,line width=0.8pt] (-1.6180339887498947,2.9021130325903073)-- (-1.,1.);
%				\draw [cackithi,line width=0.8pt] (-1.,1.)-- (1.,1.);
%				\draw [cackithi,line width=0.8pt] (1.,1.)-- (1.618033988749895,2.9021130325903064);
%				\draw [cackithi,line width=0.8pt] (1.618033988749895,2.9021130325903064)-- (0.,4.077683537175253);
%					\draw [fill=cackithi] (-1.,1.) circle (2.5pt);
%%					\draw[color=cackithi] (-1.18624795654696,0.7809397247271022) node {$A$};
%					\draw [fill=cackithi] (1.,1.) circle (2.5pt);
%%					\draw[color=cackithi] (1.209997363286399,0.7471897906449423) node {$B$};
%					\draw [fill=cackithi] (1.618033988749895,2.9021130325903064) circle (2.5pt);
%					\draw[color=cackithi] (1.8681210778885187,3.1434351104783014) node {$9$};
%					\draw [fill=cackithi] (0.,4.077683537175253) circle (2.5pt);
%					\draw[color=cackithi] (0.028749670410799504,4.4428075726414615) node {$4$};
%					\draw [fill=cackithi] (-1.6180339887498947,2.9021130325903073) circle (2.5pt);
%					\draw[color=cackithi] (-1.9624964404366394,3.0928102093550613) node {$10$};
%					\draw [fill=cackithi] (0.,2.38) circle (2.5pt);
%					\draw[color=cackithi] (-0.005000263671360479,2.012812318725942) node {$0$};
%		\end{tikzpicture}
%		\vspace*{-10pt}
%	\end{figure}
%	\textbf{\color{cackithi}B. Trường hợp thẳng hàng}
%	\vskip 0.05cm
%	Với mỗi số nguyên dương $n$, ta xét hình $L_n$ gồm $n+1$ điểm thẳng hàng và $n$ đoạn thẳng được tạo thành từ các điểm kề nhau.
%	\vskip 0.05cm
%	Ta đề xuất sự gán nhãn duyên dáng của những điểm của hình $L_4$ như sau :
%	\begin{figure}[H]
%		\vspace*{-10pt}
%		\centering
%		\captionsetup{labelformat= empty, justification=centering}
%		\begin{tikzpicture}
%			\draw[cackithi, line width=0.8pt] (0,0) -- (4,0);
%			\draw [fill=cackithi] (0,0) node[above]{$0$} circle (2.5pt);
%			\draw [fill=cackithi] (1,0) node[above]{$4$} circle (2.5pt);
%			\draw [fill=cackithi] (2,0) node[above]{$1$} circle (2.5pt);
%			\draw [fill=cackithi] (3,0) node[above]{$3$} circle (2.5pt);
%			\draw [fill=cackithi] (4,0) node[above]{$2$} circle (2.5pt);
%			
%			\draw(0.5,-0.5) node[squarednode]{$4$};
%			\draw(1.5,-0.5) node[squarednode]{$3$};
%			\draw(2.5,-0.5) node[squarednode]{$2$};
%			\draw(3.5,-0.5) node[squarednode]{$1$};
%		\end{tikzpicture}
%		\vspace*{-10pt}
%	\end{figure}
%	$1.$ Chứng minh rằng tồn tại một dán nhãn duyên dáng cho mỗi hình $L_5,L_6$ và $L_7$.
%	\vskip 0.05cm
%	$2.$ Ta chấp nhận mà không chứng minh rằng tồn tại một dán nhãn duyên dáng đối với hình $L_{2022}$ sao cho điểm ngoài cùng bên trái được gắn số $0$. Hãy mô tả sự dán nhãn này. 
%	\vskip 0.05cm
%	\textbf{\color{cackithi}C. Trường hợp đa giác}
%	\vskip 0.05cm
%	$1.$ Chứng minh rằng mọi tam giác và tứ giác đều có thể được dán nhãn một cách duyên dáng. 
%	\vskip 0.05cm
%	$2.$ Dựa vào dán nhãn duyên dáng của hình đa giác $11$ cạnh dưới đây, hãy chỉ ra một cách dán nhãn duyên dáng của đa giác $12$ cạnh.
%	\begin{figure}[H]
%		\vspace*{-5pt}
%		\centering
%		\captionsetup{labelformat= empty, justification=centering}
%		\begin{tikzpicture}[scale=0.6]
%			\draw [cackithi, line width=0.8pt] (4.,0.)-- (6.,0.);
%			\draw [cackithi,line width=0.8pt] (6.,0.)-- (7.682507065662361,1.0812816349111944);
%			\draw [cackithi,line width=0.8pt] (7.682507065662361,1.0812816349111944)-- (8.513337091666134,2.90054562562023);
%			\draw [cackithi,line width=0.8pt] (8.513337091666134,2.90054562562023)-- (8.228707415119565,4.880188509382095);
%			\draw [cackithi,line width=0.8pt] (8.228707415119565,4.880188509382095)-- (6.918985947228994,6.3916876580906115);
%			\draw [cackithi,line width=0.8pt] (6.918985947228994,6.3916876580906115)-- (5.,6.955152771773471);
%			\draw [cackithi,line width=0.8pt] (5.,6.955152771773471)-- (3.081014052771007,6.391687658090612);
%			\draw [cackithi,line width=0.8pt] (3.081014052771007,6.391687658090612)-- (1.7712925848804368,4.880188509382096);
%			\draw [cackithi,line width=0.8pt] (1.7712925848804368,4.880188509382096)-- (1.4866629083338658,2.9005456256202327);
%			\draw [cackithi,line width=0.8pt] (1.4866629083338658,2.9005456256202327)-- (2.317492934337639,1.0812816349111949);
%			\draw [cackithi,line width=0.8pt] (2.317492934337639,1.0812816349111949)-- (4.,0.);
%		
%			\draw [fill=cackithi] (4.,0.) circle (2.5pt);
%			\draw[color=cackithi] (3.868571428571432,-0.35) node {$2$};
%			\draw(5,-0.5) node[squarednode]{$7$};
%			\draw [fill=cackithi] (6.,0.) circle (2.5pt);
%			\draw[color=cackithi] (6.09714285714286,-0.35) node {$9$};
%			\draw(7,-0.1) node[squarednode]{$5$};
%			\draw [fill=cackithi] (7.682507065662361,1.0812816349111944) circle (2.5pt);
%			\draw[color=cackithi] (8.001904761904765,0.9009523809523803) node {$4$};
%			\draw(8.6,1.75) node[squarednode]{$4$};
%			\draw [fill=cackithi] (8.513337091666134,2.90054562562023) circle (2.5pt);
%			\draw[color=cackithi] (8.84,2.8057142857142856) node {$8$};
%			\draw(9,4) node[squarednode]{$3$};
%			\draw [fill=cackithi] (8.228707415119565,4.880188509382095) circle (2.5pt);
%			\draw[color=cackithi] (8.573333333333336,5.0914285714285725) node {$5$};
%			\draw(8.1,6) node[squarednode]{$2$};
%			\draw [fill=cackithi] (6.918985947228994,6.3916876580906115) circle (2.5pt);
%			\draw[color=cackithi] (7.144761904761908,6.786666666666668) node {$7$};
%			\draw(6.1,7.4) node[squarednode]{$1$};
%			\draw [fill=cackithi] (5.,6.955152771773471) circle (2.5pt);
%			\draw[color=cackithi] (5.030476190476193,7.453333333333335) node {$6$};
%			\draw(4,7.4) node[squarednode]{$6$};
%			\draw [fill=cackithi] (3.081014052771007,6.391687658090612) circle (2.5pt);
%			\draw[color=cackithi] (2.9161904761904793,6.824761904761906) node {$0$};
%			\draw(1.8,6) node[squarednode]{$11$};
%			\draw [fill=cackithi] (1.7712925848804368,4.880188509382096) circle (2.5pt);
%			\draw[color=cackithi] (1.4495238095238128,5.1485714285714295) node {$11$};
%			\draw(0.9,4) node[squarednode]{$10$};
%			\draw [fill=cackithi] (1.4866629083338658,2.9005456256202327) circle (2.5pt);
%			\draw[color=cackithi] (1.125714285714289,2.977142857142857) node {$1$};
%			\draw(1.45,1.75) node[squarednode]{$9$};
%			\draw [fill=cackithi] (2.317492934337639,1.0812816349111949) circle (2.5pt);
%			\draw[color=cackithi] (1.9638095238095272,0.8628571428571423) node {$10$};
%			\draw(3,-0.1) node[squarednode]{$8$};
%		\end{tikzpicture}
%		\vspace*{-5pt}
%	\end{figure}
%	$3.$ Xác định tính chẵn lẻ đối với trọng số của một đoạn thẳng khi mà các số được gán cho các đầu mút của nó: 
%	\vskip 0.05cm
%	$a.$ Khác nhau về tính chẵn lẻ.
%	\vskip 0.01cm
%	$b.$ Cùng tính chẵn lẻ. 
%	\vskip 0.01cm
%	$4.$ Từ đó suy ra rằng hình ngũ giác  không thể có dán nhãn duyên dáng. 
%	\vskip 0.05cm
%	\textbf{\color{cackithi}D. Một hình đa giác với số cạnh lớn}
%	\vskip 0.05cm
%	Ta ký hiệu $K_{2022}$ là hình được tạo thành từ $2022$ điểm sao cho mỗi cặp điểm bất kỳ trong chúng được nối với nhau bằng một đoạn thẳng duy nhất.
%	\vskip 0.05cm
%	$1.$ Chứng minh rằng $K_{2022}$ được tạo thành từ $2 043 231$ đoạn thẳng.
%	\vskip 0.05cm
%	$2.$ Giả sử rằng tồn tại một dán nhãn duyên dáng đối với hình $K_{2022}$.
%	\vskip 0.05cm
%	$a.$ Có bao nhiêu đoạn thẳng mang trọng số là số lẻ?
%	\vskip 0.05cm
%	$b.$ Ta ký hiệu $p$ là số điểm được dán nhãn là một số chẵn. Biểu diễn theo tham số $p$, số đoạn thẳng mà ở đó trọng số là một số lẻ.
%	\vskip 0.05cm 
%	$3.$ Chứng minh rằng hình $K_{2022}$ không thể có dán nhãn duyên dáng.
%	\vskip 0.05cm
%	\textbf{\color{cackithi}Bài $\pmb{2}$ (Dành cho thí sinh theo chương trình chuyên)}
%	\vskip 0.05cm
%	\textbf{\color{cackithi}Những số phân chia được}
%	\vskip 0.05cm
%	\textbf{\color{cackithi}Phần A}
%	\vskip 0.05cm
%	Ta nói rằng một số tự nhiên là \textit{phân chia được đơn nguyên} nếu như số đó lớn hơn hoặc bằng $3$ và viết được dưới dạng : $1+2+3+\cdots +p$, trong đó $p$ là một số tự nhiên lớn hơn hoặc bằng $2$. Ví dụ, $3$ và $10$ là những số tự nhiên phân chia được đơn nguyên bởi vì: $3 =1+2$  và $10=1+2+3+4$.
%	\vskip 0.05cm 
%	Ta nhắc lại rằng, tổng các số tự nhiên từ $1$ tới $n$ được cho bởi công thức:
%	\setlength{\abovedisplayskip}{4pt}
%	\setlength{\belowdisplayskip}{4pt} 
%	\begin{align*}
%		1+2+3+\cdots+n=\frac{n(n+1)}{2}.
%	\end{align*}
%	$1.$ $a.$ Chứng minh rằng $21$ và $136$ là những số tự nhiên phân chia được đơn nguyên.  
%	\vskip 0.05cm
%	$b.$ Số tự nhiên $1850$ có phân chia được đơn nguyên không?
%	\vskip 0.05cm
%	$2.$ Xét $a$ là một số tự nhiên lớn hơn hoặc bằng $3$. Hãy xác định điều kiện cần và đủ sao cho $a$ là một số tự nhiên phân chia được đơn~nguyên. 
%	\vskip 0.05cm
%	\textbf{\color{cackithi}Phần B}
%	\vskip 0.05cm
%	Ta nói rằng một số tự nhiên là phân chia được nếu nó có thể viết dưới dạng tổng của hai hoặc nhiều hơn các số nguyên dương liên tiếp. Ví dụ, $24$ và $25$ là những số tự nhiên phân chia được vì $24 = 7 + 8 + 9$ và $25 = 12 + 13$. Tuy nhiên $4$ không phải là số tự nhiên phân chia được vì $1 + 2 < 4 < 1 + 2 + 3$ và $2 + 3 > 4$.
%	\vskip 0.05cm
%	$1.$ Chứng minh rằng $9$ và $15$ là những số tự nhiên phân chia được nhưng $16$ thì không.
%	\vskip 0.05cm
%	$2.$ Chứng minh rằng mọi số nguyên lẻ lớn hơn hoặc bằng $3$ là phân chia được. 
%	\vskip 0.05cm
%	Xét $k$ và $q$ là những số tự nhiên với $k\ge 2$.  Đặt $S=(q+1)+(q+2)+\cdots+(q+k)$. Chứng minh rằng: $2S=k(k+1+2q)$.
%	\vskip 0.05cm
%	$4.$ Chứng minh rằng mọi lũy thừa của $2$ đều không phân chia được. 
%	 \vskip 0.05cm
%	$5.$ Chúng ta quan tâm đến những số nguyên dương chẵn và không phải là lũy thừa của $2$. Gọi $n$ là một số như thế. Ta chấp nhận rằng tồn tại duy nhất một cặp số tự nhiên $(r,m)$ trong đó $m$ là một số tự nhiên lẻ lớn hơn hoặc bằng $3$ và $r$ một số tự nhiên lớn hơn hoặc bằng $1$, sao cho  $n=2^r\times m$.
%	\vskip 0.05cm
%	$a.$ Xác định $r$ và $m$ khi $n=56$. Từ đó chỉ ra rằng $56$ là một số phân chia được và hãy viết nó dưới dạng tổng của các số nguyên dương liên tiếp.
%	\vskip 0.05cm
%	$b.$ Chứng minh rằng $44$ là phân chia được.
%	\vskip 0.05cm
%	$c.$ Chứng minh rằng mọi  số nguyên dương chẵn và không phải là lũy thừa của $2$ là phân chia được. 
%	\vskip 0.05cm
%	$6.$ Từ những kết quả trên, hãy xác định tập hợp tất cả các số tự nhiên phân chia được.
%	\vskip 0.05cm
%	\textbf{\color{cackithi}Phần C}
%	\vskip 0.05cm
%	Một số tự nhiên được gọi là phân chia được một cách duy nhất nếu số đó được biểu diễn một cách duy nhất dưới dạng tổng của hai hoặc nhiều hơn các số nguyên dương liên tiếp.
%	\vskip 0.05cm
%	$1.$ Chứng minh rằng $13$ là số phân chia được một cách duy nhất. Số $25$ có phải là số phân chia được một cách duy nhất không?
%	\vskip 0.05cm
%	$2.$ $a.$ Xét số tự nhiên $n$ là tổng của k số tự nhiên dương liên tiếp, với $k\ ge3$. Ta có thể viết $n$ dưới dạng  $n=(q+1)+(q+2)+\cdots+(q+k)$, với $q$ là số tự nhiên. Chứng minh rằng $n$ không phải là số nguyên tố.
%	\vskip 0.05cm
%	$b$. Từ đó suy ra rằng mọi số nguyên tố lớn hơn hoặc bằng $3$ là phân chia được một cách duy nhất.
%	\vskip 0.05cm
%	\textbf{\color{cackithi}Bài $\pmb{3}$ (Dành cho các thí sinh không theo chương trình chuyên)}
%	\vskip 0.05cm
%		\textbf{\color{cackithi}Số ba}
%	\vskip 0.05cm
%	Ta xây dựng một dãy số tự nhiên dựa trên quy tắc sau.
%	\vskip 0.05cm
%	\textbf{\color{cackithi}Quy tắc}
%	\vskip 0.05cm
%	Số hạng đầu tiên của dãy là $4$.
%	\vskip 0.05cm
%	Từ một số hạng, để có số hạng tiếp theo, ta thực hiện một trong những phép toán sau:
%	\vskip 0.05cm 
%	$\bullet$ Nhân số đó với $3$;
%	\vskip 0.05cm
%	$\bullet$ Nhân số với $3$ rồi cộng kết quả nhận được với $2$;
%	\vskip 0.05cm
%	$\bullet$ Nếu là số chẵn thì chia cho $2$.
%	\vskip 0.05cm
%	Nếu một trong các dãy được xây dựng theo cách này có số hạng nào đó bằng $N$, thì ta nói rằng $N$ là \textit{số có thể đạt được}.
%	\vskip 0.05cm 
%	Ví dụ, số $11$ có thể đạt được: thật vậy, ta bắt đầu từ số $4$, nhân $4$ với $3$ để được $12$, sau đó ta chia $12$ cho $2$ hai lần liên tiếp để được $3$, sau đó nhân $3$ với $3$ rồi cộng $2$ ta được kết quả là $11$.
%	\vskip 0.05cm 
%	$1.$ Chứng tỏ rằng tất cả các số tự nhiên từ $1$ đến $12$ đều có thể đạt được bằng quy tắc nêu trên. 
%	\vskip 0.05cm
%	$2.$ Chứng tỏ rằng $2022$ có thể đạt được bằng quy tắc nêu trên. 
%	\vskip 0.05cm
%	$3.$ Giả sử rằng tồn tại các số tự nhiên không thể đạt được bằng cách áp dụng quy tắc nêu trên. Gọi $m$ là số nhỏ nhất như vậy.
%	\vskip 0.05cm
%	$a.$ Chứng tỏ rằng $m$ không phải là bội của $3$.
%	\vskip 0.05cm
%	$b.$ Chứng tỏ rằng $m-2$ không phải là bội của~$3$.
%	\vskip 0.05cm
%	$c.$ Chứng tỏ rằng $m-1$ cũng không phải là bội của $3$.
%	\vskip 0.05cm
%	$d.$ Dựa vào kết quả trên, hãy đưa ra kết luận. 
%	\end{multicols}

\begin{multicols}{2}
	\vskip 0.05cm
	\textbf{\color{cackithi}Lời giải}
	\vskip 0.05cm
	\textbf{\color{cackithi}Bài $\pmb{1}$ ( Chung cho tất cả thí sinh).}
	 \vskip 0.05cm
	\textbf{\color{cackithi}Sự dán nhãn duyên dáng của một hình.}
	\vskip 0.05cm
	\textbf{\color{cackithi}Một vài ví dụ.}
	\vskip 0.05cm
	$1.$ Hình thứ nhất không phải là một dán nhãn duyên dáng bởi vì có hai trọng số giống nhau. 
	\begin{figure}[H]
		\vspace*{-10pt}
		\centering
		\captionsetup{labelformat= empty, justification=centering}
		\begin{tikzpicture}
			\draw [cackithi,line width=0.8pt] (0.,2.)-- (4.,2.);
			\draw [cackithi,line width=0.8pt] (4.,2.)-- (4.,0.);
			\draw [cackithi,line width=0.8pt] (4.,0.)-- (0.,0.);
			\draw [cackithi,line width=0.8pt] (0.,0.)-- (0.,2.);
			\draw [cackithi,line width=0.8pt] (2.,2.)-- (2.,0.);
			
			\draw [fill=cackithi] (0.,2.) circle (2.5pt);
			\draw[color=cackithi] (-0.22,2.45) node {$0$};
			\draw [fill=cackithi] (4.,2.) circle (2.5pt);
			\draw[color=cackithi] (4.24,2.45) node {$3$};
			\draw [fill=cackithi] (4.,0.) circle (2.5pt);
			\draw[color=cackithi] (4.22,-0.4) node {$5$};
			\draw [fill=cackithi] (0.,0.) circle (2.5pt);
			\draw[color=cackithi] (-0.2,-0.4) node {$7$};
			\draw [fill=cackithi] (2.,2.) circle (2.5pt);
			\draw[color=cackithi] (2.,2.45) node {$6$};
			\draw [fill=cackithi] (2.,0.) circle (2.5pt);
			\draw[color=cackithi] (1.98,-0.4) node {$1$};
			\draw (1,2.5) node[squarednode] {$6$};
			\draw (1,-0.5) node[squarednode] {$6$};
		\end{tikzpicture}
		\vspace*{-5pt}
	\end{figure}
	Hình thứ hai cũng không phải là một dán nhãn duyên dáng bởi vì có hai nhãn $7$  bằng nhau. 
	\begin{figure}[H]
		\vspace*{-10pt}
		\centering
		\captionsetup{labelformat= empty, justification=centering}
		\begin{tikzpicture}
			\draw [cackithi, line width=0.8pt] (0.,2.)-- (4.,2.);
			\draw [cackithi,line width=0.8pt] (4.,2.)-- (4.,0.);
			\draw [cackithi,line width=0.8pt] (4.,0.)-- (0.,0.);
			\draw [cackithi,line width=0.8pt] (0.,0.)-- (0.,2.);
			\draw [cackithi,line width=0.8pt] (2.,2.)-- (2.,0.);
			
			\draw [fill=cackithi] (0.,2.) circle (2.5pt);
			\draw[color=cackithi] (-0.22,2.45) node {$0$};
			\draw [fill=cackithi] (4.,2.) circle (2.5pt);
			\draw[color=cackithi] (4.24,2.45) node {$7$};
			\draw [fill=cackithi] (4.,0.) circle (2.5pt);
			\draw[color=cackithi] (4.22,-0.4) node {$4$};
			\draw [fill=cackithi] (0.,0.) circle (2.5pt);
			\draw[color=cackithi] (-0.2,-0.4) node {$7$};
			\draw [fill=cackithi] (2.,2.) circle (2.5pt);
			\draw[color=cackithi] (2.,2.45) node {$7$};
			\draw [fill=cackithi] (2.,0.) circle (2.5pt);
			\draw[color=cackithi] (1.98,-0.4) node {$2$};
		\end{tikzpicture}
		\vspace*{-10pt}
	\end{figure}
	$2.$ Hình đã cho có thể được bổ sung như sau: 
	\begin{figure}[H]
		\vspace*{-10pt}
		\centering
		\captionsetup{labelformat= empty, justification=centering}
		\begin{tikzpicture}
			\draw [cackithi,line width=0.8pt] (-1.6180339887498947,2.9021130325903073)-- (0.,2.38);
			\draw [cackithi,line width=0.8pt] (0.,2.38)-- (-1.,1.);
			\draw [cackithi,line width=0.8pt] (0.,2.38)-- (1.,1.);
			\draw [cackithi,line width=0.8pt] (0.,2.38)-- (1.618033988749895,2.9021130325903064);
			\draw [cackithi,line width=0.8pt] (0.,4.077683537175253)-- (0.,2.38);
			\draw [cackithi,line width=0.8pt] (0.,4.077683537175253)-- (-1.6180339887498947,2.9021130325903073);
			\draw [cackithi,line width=0.8pt] (-1.6180339887498947,2.9021130325903073)-- (-1.,1.);
			\draw [cackithi,line width=0.8pt] (-1.,1.)-- (1.,1.);
			\draw [cackithi,line width=0.8pt] (1.,1.)-- (1.618033988749895,2.9021130325903064);
			\draw [cackithi,line width=0.8pt] (1.618033988749895,2.9021130325903064)-- (0.,4.077683537175253);
			\draw [fill=cackithi] (-1.,1.) circle (2.5pt);
			\draw[color=cackithi] (-1.18624795654696,0.5) node[sqnode] {$3$};
			\draw (0,0.5) node[squarednode] {$2$};
			\draw (-1.5,1.7) node[squarednode] {$7$};
			\draw (-0.5,1.7) node[squarednode] {$3$};
			\draw (0.5,1.7) node[squarednode] {$1$};
			\draw (1.5,1.7) node[squarednode] {$8$};
			\draw (-1,3.5) node[squarednode] {$6$};
			\draw (-1,2.6) node[squarednode] {$10$};
			\draw (0,3.5) node[squarednode] {$4$};
			\draw (1,2.6) node[squarednode] {$9$};
			\draw (1,3.5) node[squarednode] {$5$};
			\draw [fill=cackithi] (1.,1.) circle (2.5pt);
			\draw[color=cackithi] (1.209997363286399,0.5) node[sqnode] {$1$};
			\draw [fill=cackithi] (1.618033988749895,2.9021130325903064) circle (2.5pt);
			\draw[color=cackithi] (1.8681210778885187,3.1434351104783014) node {$9$};
			\draw [fill=cackithi] (0.,4.077683537175253) circle (2.5pt);
			\draw[color=cackithi] (0.028749670410799504,4.4428075726414615) node {$4$};
			\draw [fill=cackithi] (-1.6180339887498947,2.9021130325903073) circle (2.5pt);
			\draw[color=cackithi] (-1.9624964404366394,3.0928102093550613) node {$10$};
			\draw [fill=cackithi] (0.,2.38) circle (2.5pt);
			\draw[color=cackithi] (-0.005000263671360479,2.012812318725942) node {$0$};
		\end{tikzpicture}
%		\vspace*{-5pt}
	\end{figure}
	\textbf{\color{cackithi}B. Trường hợp thẳng hàng}
	\vskip 0.05cm
	$1.$ Một dán nhãn duyên dáng của hình $L_5$	 

	\begin{figure}[H]
		\vspace*{-5pt}
		\centering
		\captionsetup{labelformat= empty, justification=centering}
		\begin{tikzpicture}
			\draw[cackithi, line width=0.8pt] (0,0) -- (5,0);
			\draw [fill=cackithi] (0,0) node[above]{$5$} circle (2.5pt);
			\draw [fill=cackithi] (1,0) node[above]{$0$} circle (2.5pt);
			\draw [fill=cackithi] (2,0) node[above]{$4$} circle (2.5pt);
			\draw [fill=cackithi] (3,0) node[above]{$1$} circle (2.5pt);
			\draw [fill=cackithi] (4,0) node[above]{$3$} circle (2.5pt);
			\draw [fill=cackithi] (5,0) node[above]{$2$} circle (2.5pt);
			
			
			\draw(0.5,-0.5) node[squarednode]{$5$};
			\draw(1.5,-0.5) node[squarednode]{$4$};
			\draw(2.5,-0.5) node[squarednode]{$3$};
			\draw(3.5,-0.5) node[squarednode]{$2$};
			\draw(4.5,-0.5) node[squarednode]{$1$};
		\end{tikzpicture}
		\vspace*{-10pt}
	\end{figure}
	Một dán nhãn duyên dáng của hình $L_6$	 

	\begin{figure}[H]
		\vspace*{-5pt}
		\centering
		\captionsetup{labelformat= empty, justification=centering}
		\begin{tikzpicture}
			\draw[cackithi, line width=0.8pt] (0,0) -- (6,0);
			\draw [fill=cackithi] (0,0) node[above]{$0$} circle (2.5pt);
			\draw [fill=cackithi] (1,0) node[above]{$6$} circle (2.5pt);
			\draw [fill=cackithi] (2,0) node[above]{$1$} circle (2.5pt);
			\draw [fill=cackithi] (3,0) node[above]{$5$} circle (2.5pt);
			\draw [fill=cackithi] (4,0) node[above]{$2$} circle (2.5pt);
			\draw [fill=cackithi] (5,0) node[above]{$4$} circle (2.5pt);
			\draw [fill=cackithi] (6,0) node[above]{$3$} circle (2.5pt);
			
			
			\draw(0.5,-0.5) node[squarednode]{$6$};
			\draw(1.5,-0.5) node[squarednode]{$5$};
			\draw(2.5,-0.5) node[squarednode]{$4$};
			\draw(3.5,-0.5) node[squarednode]{$3$};
			\draw(4.5,-0.5) node[squarednode]{$2$};
			\draw(5.5,-0.5) node[squarednode]{$1$};
		\end{tikzpicture}
		\vspace*{-10pt}
	\end{figure}
	Sự dán nhãn duyên dáng của hình $L_7$	 

	\begin{figure}[H]
		\vspace*{-5pt}
		\centering
		\captionsetup{labelformat= empty, justification=centering}
		\begin{tikzpicture}
			\draw[cackithi, line width=0.8pt] (0,0) -- (7,0);
			\draw [fill=cackithi] (0,0) node[above]{$7$} circle (2.5pt);
			\draw [fill=cackithi] (1,0) node[above]{$0$} circle (2.5pt);
			\draw [fill=cackithi] (2,0) node[above]{$6$} circle (2.5pt);
			\draw [fill=cackithi] (3,0) node[above]{$1$} circle (2.5pt);
			\draw [fill=cackithi] (4,0) node[above]{$5$} circle (2.5pt);
			\draw [fill=cackithi] (5,0) node[above]{$2$} circle (2.5pt);
			\draw [fill=cackithi] (6,0) node[above]{$4$} circle (2.5pt);
			\draw [fill=cackithi] (7,0) node[above]{$3$} circle (2.5pt);
			
			
			\draw(0.5,-0.5) node[squarednode]{$7$};
			\draw(1.5,-0.5) node[squarednode]{$6$};
			\draw(2.5,-0.5) node[squarednode]{$5$};
			\draw(3.5,-0.5) node[squarednode]{$4$};
			\draw(4.5,-0.5) node[squarednode]{$3$};
			\draw(5.5,-0.5) node[squarednode]{$2$};
			\draw(6.5,-0.5) node[squarednode]{$1$};
		\end{tikzpicture}
		\vspace*{-10pt}
	\end{figure}
	$2.$ Tương tự như các dán nhãn của hình $L_4$ và $L_6$ phía trên, ta có thể dán nhãn hình $L_{2022}$ như sau: ta đánh số các điểm từ trái qua phải dựa vào dãy sau: 
	\begin{align*}
		&0,2022,1,2021,2,2020,3,2019,4,\\
		&2018 \ldots,1000,1012,1011
	\end{align*}
	Với cách dán nhãn trên, ta nhận được các trọng số từ trái qua phải là: $2022 ,$ $2021,$ $\ldots,4,3,2,1$. Đó là một dán nhãn duyên dáng của hình $L_{2022}$.
	\vskip 0.05cm 
	\textbf{\color{cackithi}C. Trường hợp đa giác}
	\vskip 0.05cm
	$1.$ Ta có thể dán nhãn tam giác và tứ giác một cách duyên dáng như sau: 
	\begin{figure}[H]
		\centering
		\vspace*{-5pt}
		\captionsetup{labelformat= empty, justification=centering}
		\begin{tikzpicture}[scale=0.7]
			\draw [cackithi, line width=0.8pt] (1.223091301970301,2.0066510197535226)-- (0,0);
			\draw [cackithi,line width=0.8pt] (0,0)-- (4,0);
			\draw [cackithi,line width=0.8pt] (4,0)-- (1.223091301970301,2.0066510197535226);
			\draw [cackithi,line width=0.8pt] (6,0)-- (5,2);
			\draw [cackithi,line width=0.8pt] (5,2)-- (8,3);
			\draw [cackithi,line width=0.8pt] (8,3)-- (9,0);
			\draw [cackithi,line width=0.8pt] (6,0)-- (9,0);

				\draw [fill=cackithi] (1.223091301970301,2.0066510197535226) circle (2.5pt);
				\draw[color=cackithi] (1.2043242804043022,2.522744112818487) node {$0$};
				\draw [fill=cackithi] (0,0) circle (2.5pt);
				\draw[color=cackithi] (-0.259503401743597,-0.14217294955332546) node {$1$};
				\draw[color=cackithi] (0.4161093746323564,1.3779557972925676) node[squarednode] {$1$};
				\draw [fill=cackithi] (4,0) circle (2.5pt);
				\draw[color=cackithi] (4.300882838794089,-0.17970699268532278) node {$3$};
				\draw[color=cackithi] (1.8048689705162608,0.4771387621246309) node[squarednode] {$2$};
				\draw[color=cackithi] (2.8558221782121884,1.4717909051225608) node[squarednode] {$3$};
				\draw [fill=cackithi] (6,0) circle (2.5pt);
				\draw[color=cackithi] (5.708409456243992,-0.16093997111932415) node {$2$};
				\draw [fill=cackithi] (5,2) circle (2.5pt);
				\draw[color=cackithi] (4.694990291680062,2.4664430481204906) node {$4$};
				\draw[color=cackithi] (5.839778607205983,1.3591887757265688) node[squarednode] {$2$};
				\draw [fill=cackithi] (8,3) circle (2.5pt);
				\draw[color=cackithi] (8.148122259823824,3.4047941264204247) node {$0$};
				\draw[color=cackithi] (6.721828620807923,2.3538409187244986) node[squarednode] {$4$};
				\draw [fill=cackithi] (9,0) circle (2.5pt);
				\draw[color=cackithi] (9.23660951065175,-0.23600805738331884) node {$3$};
				\draw[color=cackithi] (8.78620099306778,1.8283643148765356) node[squarednode] {$3$};
				\draw[color=cackithi] (7.510043526579868,0.4771387621246309) node[squarednode] {$1$};
		\end{tikzpicture}
		\vspace*{-10pt}
	\end{figure}
	$2.$ Bằng cách thêm đỉnh số $12$ như hình dưới vào một đa giác $11$ cạnh cho trước ta nhận được một dán nhãn duyên dáng của đa giác $12$ cạnh. 
	\begin{figure}[H]
		\centering
%		\vspace*{-5pt}
		\captionsetup{labelformat= empty, justification=centering}
		\begin{tikzpicture}
			\draw [cackithi, line width=0.8pt] (-3,3)-- (-4,1);
			\draw [cackithi,dashed, line width=0.8pt] (-4,1)-- (-1,2);
			\draw [cackithi,line width=0.8pt] (-3,3)-- (-1,2);
		
				\draw [fill=cackithi] (-3,3) circle (2.5pt);
				\draw[color=cackithi] (-3.018255571945407,3.4610951911184205) node[sqnode] {$12$};
				\draw [fill=cackithi] (-4,1) circle (2.5pt);
				\draw[color=cackithi] (-4.294413038433319,0.8524791934446044) node {$0$};
				\draw[color=cackithi] (-3.6938683483213604,2.0) node[squarednode] {$12$};
				\draw [fill=cackithi] (-1,2) circle (2.5pt);
				\draw[color=cackithi] (-0.6723778761955685,1.978500487404525) node {$6$};
				\draw[color=cackithi] (-1.986069385815478,2.8) node[squarednode] {$6$};
		\end{tikzpicture}
		\vspace*{-10pt}
	\end{figure}
	$3a.$ Nếu hai đỉnh kề nhau khác tính chẵn lẻ thì hiệu của chúng là một số lẻ, do đó trọng số là số lẻ. 
	\vskip 0.05cm
	$3b.$ Hoàn toàn tương tự như trên, nếu hai đỉnh kề nhau có cùng tính chẵn lẻ thì trọng số của đoạn thẳng nối hai đỉnh đó là một số chẵn.  
	\vskip 0.05cm
	$4.$ Giả sử phản chứng rằng tồn tại dán nhãn duyên dáng đối với hình ngũ giác. Khi đó trọng số các cạnh sẽ là các số tự nhiên từ $1$ tới $5$, trong đó có $3$ số lẻ và $2$ số chẵn. Đối với $3$ cạnh có trọng số lẻ thì các đỉnh liên kết phải khác tính chẵn lẻ, nếu không trọng số sẽ là số chẵn theo chứng minh trên. Ta có hai trường hợp sau:
	\vskip 0.05cm 
	Trường hợp ${1}$: $3$ cạnh trọng số lẻ kề nhau.
	\begin{figure}[H]
		\centering
		\vspace*{-5pt}
		\captionsetup{labelformat= empty, justification=centering}
		\begin{tikzpicture}[scale=0.75]
			\draw [line width=0.8pt,color=cackithi] (-1,0)-- (1,0);
			\draw [line width=0.8pt,color=cackithi] (1,0)-- (1.618033988749895,1.9021130325903064);
			\draw [line width=0.8pt,color=cackithi] (1.618033988749895,1.9021130325903064)-- (0,3.077683537175253);
			\draw [line width=0.8pt,color=cackithi] (0,3.077683537175253)-- (-1.6180339887498947,1.9021130325903073);
			\draw [line width=0.8pt,color=cackithi] (-1.6180339887498947,1.9021130325903073)-- (-1,0);
			\draw [line width=0.8pt,color=cackithi] (4,0)-- (6,0);
			\draw [line width=0.8pt,color=cackithi] (6,0)-- (6.618033988749895,1.9021130325903064);
			\draw [line width=0.8pt,color=cackithi] (6.618033988749895,1.9021130325903064)-- (5,3.077683537175253);
			\draw [line width=0.8pt,color=cackithi] (5,3.077683537175253)-- (3.381966011250105,1.9021130325903073);
			\draw [line width=0.8pt,color=cackithi] (3.381966011250105,1.9021130325903073)-- (4,0);
		
				\draw [fill=cackithi] (-1,0) circle (2.5pt);
				\draw[color=cackithi] (-1.291689587873526,-0.6) node[squarednode] {L};
				\draw(0,0)node[below] {Chẵn};
				\draw [fill=cackithi] (1,0) circle (2.5pt);
				\draw[color=cackithi] (1.2418583235362997,-0.6) node[squarednode] {L};
				\draw(1.5,1)node[below] {Lẻ};
				\draw [fill=cackithi] (1.618033988749895,1.9021130325903064) circle (2.5pt);
				\draw[color=cackithi] (1.9,1.3) node[squarednode] {C};
				\draw(1,3)node[below] {Lẻ};
				\draw [fill=cackithi] (0,3.077683537175253) circle (2.5pt);
				\draw[color=cackithi] (0.022001921746383594,3.7) node[squarednode] {L};
				\draw(-1,3)node[below] {Lẻ};
				\draw [fill=cackithi] (-1.6180339887498947,1.9021130325903073) circle (2.5pt);
				\draw[color=cackithi] (-1.929768321117482,2.5) node[squarednode] {C};
				\draw(-1.3,1)node[below] {Chẵn};
				\draw [fill=cackithi] (4,0) circle (2.5pt);
				\draw[color=cackithi] (3.85047432121012,-0.6) node[squarednode] {C};
				\draw(5,0)node[below] {Chẵn};
				\draw(6.6,0.9)node[below] {Lẻ};
				\draw(6,3)node[below] {Lẻ};
				\draw(4,3)node[below] {Lẻ};
				\draw(3,1)node[below] {Chẵn};
				\draw [fill=cackithi] (6,0) circle (2.5pt);
				\draw[color=cackithi] (6.140050952261962,-0.6) node[squarednode] {C};
				\draw [fill=cackithi] (6.618033988749895,1.9021130325903064) circle (2.5pt);
				\draw[color=cackithi] (6.909498836467909,1.2224717677625083) node[squarednode] {L};
				\draw [fill=cackithi] (5,3.077683537175253) circle (2.5pt);
				\draw[color=cackithi] (5.032796679868039,3.7) node[squarednode] {C};
				\draw [fill=cackithi] (3.381966011250105,1.9021130325903073) circle (2.5pt);
				\draw[color=cackithi] (3.081026437004173,2.5) node[squarednode] {L};
		\end{tikzpicture}
		\vspace*{-10pt}
	\end{figure}
	Trường hợp ${2}$: $2$ cạnh trọng số lẻ kề nhau liền kề với một cạnh có trọng số chẵn. 
	\begin{figure}[H]
		\centering
		\vspace*{-5pt}
		\captionsetup{labelformat= empty, justification=centering}
		\begin{tikzpicture}[scale=0.75]
			\draw [line width=0.8pt,color=cackithi] (-1,0)-- (1,0);
			\draw [line width=0.8pt,color=cackithi] (1,0)-- (1.618033988749895,1.9021130325903064);
			\draw [line width=0.8pt,color=cackithi] (1.618033988749895,1.9021130325903064)-- (0,3.077683537175253);
			\draw [line width=0.8pt,color=cackithi] (0,3.077683537175253)-- (-1.6180339887498947,1.9021130325903073);
			\draw [line width=0.8pt,color=cackithi] (-1.6180339887498947,1.9021130325903073)-- (-1,0);
			\draw [line width=0.8pt,color=cackithi] (4,0)-- (6,0);
			\draw [line width=0.8pt,color=cackithi] (6,0)-- (6.618033988749895,1.9021130325903064);
			\draw [line width=0.8pt,color=cackithi] (6.618033988749895,1.9021130325903064)-- (5,3.077683537175253);
			\draw [line width=0.8pt,color=cackithi] (5,3.077683537175253)-- (3.381966011250105,1.9021130325903073);
			\draw [line width=0.8pt,color=cackithi] (3.381966011250105,1.9021130325903073)-- (4,0);
			
			\draw [fill=cackithi] (-1,0) circle (2.5pt);
			\draw[color=cackithi] (-1.291689587873526,-0.6) node[squarednode] {C};
			\draw(0,0)node[below] {Lẻ};
			\draw [fill=cackithi] (1,0) circle (2.5pt);
			\draw[color=cackithi] (1.2418583235362997,-0.6) node[squarednode] {L};
			\draw(1.5,1)node[below] {Chẵn};
			\draw [fill=cackithi] (1.618033988749895,1.9021130325903064) circle (2.5pt);
			\draw[color=cackithi] (1.9,1.3) node[squarednode] {L};
			\draw(1,3)node[below] {Lẻ};
			\draw [fill=cackithi] (0,3.077683537175253) circle (2.5pt);
			\draw[color=cackithi] (0.022001921746383594,3.7) node[squarednode] {C};
			\draw(-1,3)node[below] {Lẻ};
			\draw [fill=cackithi] (-1.6180339887498947,1.9021130325903073) circle (2.5pt);
			\draw[color=cackithi] (-1.929768321117482,2.5) node[squarednode] {L};
			\draw(-1.3,1)node[below] {Chẵn};
			\draw [fill=cackithi] (4,0) circle (2.5pt);
			\draw[color=cackithi] (3.85047432121012,-0.6) node[squarednode] {L};
			\draw(5,0)node[below] {Lẻ};
			\draw(6.6,0.9)node[below] {Chẵn};
			\draw(6,3)node[below] {Lẻ};
			\draw(4,3)node[below] {Lẻ};
			\draw(3,1)node[below] {Chẵn};
			\draw [fill=cackithi] (6,0) circle (2.5pt);
			\draw[color=cackithi] (6.140050952261962,-0.6) node[squarednode] {C};
			\draw [fill=cackithi] (6.618033988749895,1.9021130325903064) circle (2.5pt);
			\draw[color=cackithi] (6.909498836467909,1.2224717677625083) node[squarednode] {C};
			\draw [fill=cackithi] (5,3.077683537175253) circle (2.5pt);
			\draw[color=cackithi] (5.032796679868039,3.7) node[squarednode] {L};
			\draw [fill=cackithi] (3.381966011250105,1.9021130325903073) circle (2.5pt);
			\draw[color=cackithi] (3.081026437004173,2.5) node[squarednode] {C};
		\end{tikzpicture}
		\vspace*{-10pt}
	\end{figure}
	Khi đó sẽ tồn tại hai đỉnh được dán nhãn khác tính chẵn lẻ nhưng lại cho trọng số là số chẵn như minh họa phía trên. Điều này mâu thuẫn với tính chất đã chứng minh ở phần trước. Do vậy, không tồn tại bất cứ dán nhãn duyên dáng cho hình ngũ giác.
	\vskip 0.05cm 
	\textbf{\color{cackithi}D. Một hình đa giác với số cạnh lớn}
	\vskip 0.05cm
	$1.$ Số các đoạn thẳng bằng số cách chọn ra $2$ điểm từ $2021$ điểm, nghĩa là 
	$\dfrac{1}{2}\times 2022\times 2021=2 043 231$. Vậy, hình $K_{2022}$ được tạo thành từ $2043231$ đoạn thẳng. 
	\vskip 0.05cm
	$2a.$ Số đoạn thẳng mang trọng số lẻ chính là số những số tự nhiên lẻ của tập hợp $\{1,2,...,2043231\}$, nghĩa là bằng  $1021616$.
	\vskip 0.05cm 
	$2b.$ Vì có $p$ điểm được dán nhãn là số chẵn nên số điểm được dán nhãn là số lẻ là $2022-p$. Số những đoạn thẳng có trọng số lẻ chính là số cặp điểm được dán nhãn khác nhau về tính chẵn lẻ, do đó có tất cả  $p(2022-p)$ đoạn.
	\vskip 0.05cm
	$3.$ Giả sử hình  $K_{2022}$ có một dán nhãn duyên dáng. Khi đó có $1 021 616$ đoạn thẳng mang trọng số lẻ. Suy ra tồn tại một số tự nhiên $p$ sao cho: $p(2022-p)=1 021 616$, nghĩa là $p^2-2022p+1 021 616=0$. Phương trình này không có nghiệm nguyên, mâu thuẫn. 
	\vskip 0.05cm
	\textbf{\color{cackithi}Bài $\pmb{2}$ ( Dành cho thí sinh theo chương trình chuyên)} 
	\vskip 0.05cm
	\textbf{\color{cackithi}Phần A}
	\vskip 0.05cm
	$1a$. Các số $21$ và $136$ là phân chia được đơn nguyên vì: $21=1+2+3+4+5+6$ và $136=1+2+3+4+5+6+7+8+9+10+11+12+13+14+15+16$.
	\vskip 0.05cm 
	$1b.$ Nếu $1850$ là phân chia được đơn nguyên thì tồn tại số tự nhiên $n$ sao cho: 
	\begin{align*}
		1+2+3+\cdots+n=1850.
	\end{align*}
	Suy ra 
	\begin{align*}
		\frac{n(n+1)}{2} = 1850.
	\end{align*}
	Hay, $n^2+n-3700=0$. Phương trình bậc hai này không có nghiệm nguyên, mâu thuẫn. Do đó $1850$ không phải là số phân chia được đơn nguyên. 
	\vskip 0.05cm
	$2.$ Số tự nhiên $a$ lớn hơn hoặc bằng $3$ là một số phân chia được đơn nguyên khi và chỉ khi phương trình: $n^2+n-2a=0$ có ít nhất một nghiệm nguyên dương. Điều đó có nghĩa là biệt thức $\Delta=1+8a$ là một số chính phương và ít nhất một trong hai nghiệm \linebreak$x_1=\dfrac{-1-\sqrt{1+ 8a}}{2}$ và $x_2=\dfrac{-1+\sqrt{1+ 8a}}{2}$ là nguyên dương. Từ đó ta suy ra rằng điều kiện cần và đủ để $a$ là số phân chia được đơn nguyên là $1+8a$ là một số chính phương.
	\vskip 0.05cm  
	\textbf{\color{cackithi}Phần B}
	\vskip 0.05cm
	$1.$ Các số $9$ và $15$ là phân chia được vì $9=4+5$ và $15=7+8=4+5+6=1+2+3+4+5$. Tuy nhiên số $16$ thì không phân chia được vì:
	\begin{align*}
		&1\!+\!2\!+\!3\!+\!4\!+\!5\!<\!16\!<\!1\!+\!2\!+\!3\!+\!4\!+\!5\!+\!6;\\
		&2+3+4+5<16<2+3+4+5+6;\\
		&3+4+5<16<3+4+5+6;\\
		&4+5+6<16<4+5+6+7;\\
		&5\!+\!6\!<\!16\!<\!5\!+\!6\!+\!7; 6\!+\!7\!<\!16\!<\!6\!+\!7\!+\!8;\\
		&7+8<16<7+8+9 \text{ và } 8+9>16.
	\end{align*}
	$2.$ Gọi $n$ là số tự nhiên là lẻ và lớn hơn hoặc bằng $3$. Đặt $n=2k+1$. Khi đó $n=k+(k+1)$ do đó là phân chia được.
	\vskip 0.05cm  
	$3.$ $S=(q+1)+(q+2)+\cdots+(q+k)=(q+q+\cdots+q)+(1+2+\cdots+k)=kq+\dfrac{k(k+1)}{2}$.
	\vskip 0.05cm
	Từ đó suy ra: $2S=2kq+k(k+1)=k(2q+k+1)$. 
	\vskip 0.05cm
	$4.$ Giả sử $N=2^p$ là một lũy thừa của $2$ và là phân chia được. Theo kết câu trên, tồn tại các số tự nhiên $k$ và $q$ lớn hơn hoặc bằng $2$ sao cho: $2N=k(2q+k+1)$. Điều này là vô lý vì vế trái là một luỹ thừa của $2$ còn vế phải là tích của một số chẵn và một số lẻ lớn hơn $1$. 
	\vskip 0.05cm
	$5a.$ Ta có $56=2^3\times7$ nên $r=3$ và $m=7$. Hơn nữa $2\times 56=2^4\times7=7(2\times4+7+1)$. Do đó $56$ được viết dưới dạng tổng được định nghĩa ở ý $3)$ phần $B)$ với $k=7$ và $q=4$. Cụ thể hơn $56=5+6+7+8+9+10+11$, từ đó suy ra $56$ là số tự nhiên phân chia được. 
	\vskip 0.05cm
	$5b.$ Tương tự như trên $2\times 44=8\times11=8(2\times1+8+1)$. Do đó $44=2+3+4+5+6+7+8+9$. Ta kết luận rằng $44$ là số tự nhiên phân chia được.
	\vskip 0.05cm 
	$5c.$ Gọi $n$ là một số tự nhiên dương chẵn và không phải là lũy thừa của $2$. Đặt $n=2^r\times m$, với $m$ là một số nguyên lẻ lớn hơn hoặc bằng $3$ và $r$ một số nguyên dương. Ta suy ra $2n=2^{r+1}\times m$. Ta xét hai trường hợp sau. 
	\vskip 0.05cm
	Trường hợp $1$: Nếu $m>2^{r+1}$, tức là $m \ge 2^{r+1}+1$ và vì $m$ là một số tự nhiên lẻ nên ta suy ra tồn tại một số tự nhiên $l\ge 0$ sao cho $m=2^{r+1}+1+2l$. Khi đó $2n=2^{r+1}(2l+2^{r+1}+1)$. Do đó $n$ được viết dưới dạng tổng được định nghĩa ở ý $3)$ phần $B)$ với $k=2^{r+1}$ và $q=l$. Hay nói cách khác $n$ là số tự nhiên phân chia được. 
	\vskip 0.05cm
	Trường hợp $2$: Nếu $m<2^{r+1}$ tức là $m+1 \le 2^{r+1}$ và vì $2^{r+1}$ là một số tự nhiên chẵn nên ta suy ra tồn tại một số tự nhiên $l \ge 0$ sao cho $2^{r+1}=m+1+2l$. Khi đó $2n=m(2l+m+1)$. Do đó $n$ được viết dưới dạng tổng được định nghĩa ở ý $3)$ phần $B)$ với $k=m$ và $q=l$. Ta kết luận rằng n là số tự nhiên phân chia được.  
	\vskip 0.05cm
	Lưu ý rằng trường hợp $m=2^{r+1}$ không thể xảy ra vì $m$ là số lẻ. 
	\vskip 0.05cm
	$6.$ Từ những kết quả nhận được ở câu hỏi $2)$ và câu hỏi $5)$ ta suy ra rằng tập hợp những số tự nhiên phân chia được gồm những số tự nhiên lẻ lớn hơn hoặc bằng $3$ và những số tự nhiên chẵn không  viết được dưới dạng lũy thừa của $2$. 
	\vskip 0.05cm
	\textbf{\color{cackithi}Phần C}
	\vskip 0.05cm
	$1.$ $13$ là số tự nhiên lẻ lớn hơn $3$, nên theo kết quả trên $13$ là số tự nhiên phân chia được, hơn nữa $2\times 13=2(2\times 5+2+1)$ nên $13$ được viết dưới dạng tổng được định nghĩa ở ý $3)$ phần $B)$ với $k=2$ và $q=5$. Tức là $13=(5+1)+(5+2)$. Giả sử tồn tại một biểu diễn khác của $13$, ta suy ra tồn tại những số tự nhiên $k'\ ge 2$ và $q'$ sao cho $13=(q'+1)+(q'+2)+\cdots+(q'+k')$. Theo kết quả phần $B)$, ta có $2\times 13=k'(2q'+k'+1)$. Vì $2q'+k'+1>k'$ nên ta suy ra $k'<13$. Vì $13$ là số nguyên tố, nên ta suy ra $k'\ge 2$ là ước của $2$. Hay nói cách khác $k'=2=k$. Thay vào đẳng thức ta được $q'=5=q$. Do đó $13$ là số tự nhiên phân chia được một cách duy nhất. Tương tự $25$ là số tự nhiên phân chia được, tuy nhiên $2\times25=2\times(2\times11+2+1)=5(2\times2+5+1)$, nên theo kết quả ở phần $B)$, số tự nhiên $25$ có thể được biểu diễn dưới dạng tổng theo $2$ cách $25=12+13$ và $25=3+4+5+6+7$ nên nó không phải là số tự nhiên phân chia được một cách duy nhất.
	\vskip 0.05cm 
	$2a.$ Ta có $n=(q+1)+(q+2)+\cdots+(q+k)=k\times q+\dfrac{k(k+1)}{2}$. Nếu $k$ là số tự nhiên chẵn thì $\dfrac{k}{2}$ là một số tự nhiên, do đó $n=\dfrac{k}{2}(2q+k+1)$. Nếu $k$ là một số tự nhiên lẻ thì $\dfrac{k}{2}$ là một số tự nhiên, do đó $n=k(q+\dfrac{k+1}{2})$. Từ đó ta kết luận rằng $n$ không phải là số nguyên tố.
	\vskip 0.05cm 
	$2b$. Gọi $p$ là tố lớn hơn hoặc bằng $3$, vì $p$ là số lẻ nên theo kết quả phần $B)$ $p$ là số tự nhiên phân chia được. Hơn nữa $p=(q+1)+(q+2)$ với $q=\dfrac{p-3}{2}$. Để  ý rằng $q$ là một số tự nhiên vì $p$ là một số lẻ lớn hơn hoặc bằng $3$. Ta sẽ chứng minh biểu diễn này là duy nhất. Tương tự câu $1)$ giả sử tồn tại một biểu diễn khác của $13$, ta suy ra tồn tại những số tự nhiên $k'\ge 2$ và $q'$ sao cho $p=(q'+1)+(q'+2)+\cdots+(q'+k')$. Theo kết quả phần $B)$, ta có $2\times p=k'(2q'+k'+1)$. Vì $2q'+k'+1>k'$ nên ta suy ra $k'<p$. Vì $p$ là số nguyên tố, nên ta suy ra $k'$ là ước của $2$. Hay nói cách khác $k'=2$. Thay vào đẳng thức ta được $q'=q$. Ta kết luận rằng mọi số nguyên tố lớn hơn hoặc bằng $3$ đều phân chia được một cách duy nhất. 
	\vskip 0.05cm
	\textbf{\color{cackithi}Bài $\pmb{3}$ (Dành cho các thí sinh không theo chương trình chuyên)} 
	\vskip 0.05cm
		\textbf{\color{cackithi}Số ba}
	\vskip 0.05cm
	$1.$ Dựa vào những sơ đồ dưới đây, ta có thể khẳng định rằng cả các số tự nhiên từ $1$ đến $12$ đều có thể đạt được bằng quy tắc nêu trên. 
	\[\begin{tikzcd}[column sep = 1.35em]
		4 \arrow{r}{:\ 2}  & 2 \arrow{r}{:\ 2}& 1, &4 \arrow{r}{:\ 2}  & 2,\\[-3ex]
		4 \arrow{r}{:\ 2}  & 2 \arrow{r}{\times 3}& 6  \arrow{r}{:\  2} & 3, & 4,\\[-3ex]
		4 \arrow{r}{:\ 2}  & 2 \arrow{r}{:\ 2} &1 \arrow{r}{\times 3}& 3  \arrow{r}{+2} & 5,\\[-3ex]
		4 \arrow{r}{:\ 2}  & 2 \arrow{r}{\times 3}& 6,\\[-3ex]
		4 \arrow{r}{\times  3}  & 12 \arrow{r}{+ 2}& 14\arrow{r}{: \ 2 }& 7,\\[-3ex]
		4 \arrow{r}{:\  2}  & 2 \arrow{r}{\times 3}& 6 \arrow{r}{+2 }& 8, \\[-3ex]
		4 \arrow{r}{:\ 2}  & 2 \arrow{r}{:\ 2} &1 \arrow{r}{\times 3}& 3  \arrow{r}{\times 3} & 9, \\[-3ex]
		4 \arrow{r}{:\ 2}  & 2 \arrow{r}{\times 3} &6 \arrow{r}{\times 3}& 18  \arrow{r}{+ 2} & 20 \arrow{r}{:\ 2} & 10,\\[-3ex]
		4 \arrow{r}{:\ 2}  & 2 \arrow{r}{:\ 2} &1 \arrow{r}{\times 3}& 3  \arrow{r}{\times 3} & 9 \arrow{r}{+ 2} & 11,\\[-3ex]
		4  \arrow{r}{\times 3} &12.
	\end{tikzcd}\]
	$2.$ Dựa vào kết quả trên, ta có thể thực hiện các phép toán sao cho kết quả là $8$. Sau đó ta tiếp tục áp dụng liên tiếp các phép toán sau:
	 \[
	 \begin{tikzcd}[column sep = 1.35em]
	 	8 \arrow{r}{\times 3} & 24\arrow{r}{\times 3} & 72 \arrow{r}{+ 2} & 74 \arrow{r}{\times 3}& 222\\[-3ex]
	 	222 \arrow{r}{+ 2} & 224 \arrow{r}{\times 3} & 672 \arrow{r}{+ 2} & 674 \arrow{r}{\times 3} & 2022.
	 \end{tikzcd}
	 \]
	Ta kết luận rằng $2022$ là số tự nhiên có thể đạt được theo các quy tắc đã nêu. 
	
	$3a$. Giả sử phản chứng rằng $m$ là bội của $3$. Đặt $m=3a$. Do $m$ là số không đạt được nhỏ nhất. nhỏ nhất nên $a$ là số đạt được. Thế nhưng khi đó, ta chỉ cần áp dụng thêm phép toán Nhân $3$  với kết quả $a$ để đạt được $m$. Hay nói cách khác $m$ là số tự nhiên đạt được, mâu thuẫn. Chứng tỏ rằng giả sử phản chứng là sai, nói cách khác $m$ không phải là bội của $3$.
	\vskip 0.05cm 
	$3b$. Giả sử $m-2$ là bội của $3$, đặt $m=3b+2$. Do $m$ là số không đạt được nhỏ nhất nên $b$ là đạt được. Khi đó, chỉ cần áp dụng thêm $2$ phép toán liên tiếp Nhân $3$  rồi Cộng $2$ với từ $b$ ta thu được $m$. Hay nói cách khác $m$ là số tự nhiên đạt được, mâu thuẫn. Vậy $m-2$ không phải là bội của $3$. 
	\vskip 0.05cm
	$3c.$ Nếu $m-1$ là bội của $3$, thì tồn tại số tự nhiên dương $c$ sao cho $m=3c+1>c$. Ta suy ra $2m=3\times 2c+2$. Nếu $2c$ là số tự nhiên không đạt được bằng cách áp dụng các quy tắc như trên, thì vì $m$ là nhỏ nhất trong những số không thể đạt được nên $m\ le 2c$, hay $3c+1 \le 2c \Leftrightarrow +1 \le 0$. Điều này mâu thuẫn với điều kiện $c$ là số tự nhiên. Ngược lại, nếu $2c$ là số tự nhiên đạt được, ta áp dụng thêm $3$ phép toán liên tiếp  Nhân $3$  rồi Cộng $2$  rồi  Chia $2$ với kết quả $2c$ ta thu được $m$. Hay nói cách khác $m$ cũng là số tự nhiên đạt được. Điều này mâu thuẫn với định nghĩa của $m$. Chứng tỏ rằng $m-1$ không phải là bội của $3$.
	\vskip 0.05cm
	$3d.$ Vì trong ba số tự nhiên liên tiếp luôn có một số là bội của $3$. Nên dựa vào những kết quả trên, nếu tồn tại những số tự nhiên không đạt được bằng cách áp dụng các quy tắc đã nêu với $m$ là số tự nhiên nhỏ nhất trong số chúng, khi đó $m-2,m-1$ và $m$ đều không phải là bội của $3$. Điều này mâu thuẫn với tính chất đã nêu. Chứng tỏ, mọi số tự nhiên dương đều đạt được bằng cách áp dụng các quy tắc đã nêu. 
	\vskip 0.05cm
	\textbf{\color{cackithi}\color{cackithi}Tài liệu tham khảo}
	\vskip 0.05cm
	[$1$] Les Olympiades nationales de mathématiques | Ministère de l'Education Nationale et de la Jeunesse
	\vskip 0.05cm
	[$2$] https://www.freemaths.fr/annales-olym\\piades-mathematiques-premieres-scientifiqu\\es-s/nationales/2022
\end{multicols}
\vspace*{-10pt}
\rule{1\linewidth}{0.1pt}
\begin{center}
	\textbf{\LARGE\color{cackithi}LỜI GIẢI, ĐÁP ÁN}
\end{center}
\begin{multicols}{2}
	Sau đó, hai máy bay có đầy bình tiếp tục hành trình của mình còn chiếc còn $1/2$ bình quay trở lại sân bay ban đầu. 
	\vskip 0.05cm
	Sau khi tiếp tục đi được $1/8$ vòng Trái Đất, nghĩa là được $1/4$ vòng kể từ điểm xuất phát, chiếc còn lại chuyển $1/4$ bình nhiên liệu cho Pi. Khi này Pi có đầy bình và chiếc còn lại có $1/4$ bình, vừa đủ để quay trở lại. Bây giờ, Pi tiếp tục hành trình của mình còn chiếc máy bay kia quay trở lại sân bay ban đầu. Với đầy bình, Pi có thể đi được $3/4$ vòng Trái Đất kể từ điểm xuất phát. 
	\vskip 0.05cm
	Máy bay đầu tiên, với đầy bình bay theo hướng ngược lại (so với hướng ban đầu) để gặp máy bay của Pi ở vị trí $3/4$ vòng Trái Đất, khi đó máy bay đầu tiên còn $1/2$ bình sẽ tiếp cho máy bay của Pi $1/4$ bình, sao cho cả hai máy bay có $1/4$ bình, đủ để đi đến vị trí $7/8$ vòng Trái Đất. Khi hai chiếc máy bay này gặp nhau, chiếc còn lại bắt đầu xuất phát từ sân bay ban đầu với đầy bình và bay theo hướng ngược lại để gặp máy bay của Pi và chiếc thứ hai ở vị trí $7/8$ vòng Trái Đất và tiếp cho mỗi chiếc $1/4$ bình. Khi này, mỗi chiếc có đúng $1/4$ bình, vừa đủ để bay về sân bay ban đầu.   
	\vskip 0.05cm
	\textbf{\color{cackithi}Góc cờ}
	\vskip 0.05cm
	\textbf{\color{cackithi}Bài $\pmb{1}$.
				$\pmb{1}$.Vd$\pmb{4}$ Vb$\pmb{4}$ $\pmb{2}$.Xg$\pmb{2}$ Mh$\pmb{3}$} [$2$...Md$1$ $3$.Xd$2$]
	\vskip 0.05cm
	\textbf{\color{cackithi}$\pmb{3}$.Ke$\pmb{3}$} Mã đen bị bắt.
	\vskip 0.05cm
	\textbf{\color{cackithi}Bài $\pmb{2}$. 
				$\pmb{1}$.Vc$\pmb{4}$ Vc$\pmb{6}$} [$1$...Mh$3$ $2$.Xg$4$ Mf$2$ $3$.Xh$4$]
	\vskip 0.05cm
	\textbf{\color{cackithi}$\pmb{2}$.Xh$\pmb{4}$ Vd$\pmb{6}$ $\pmb{3}$.Vd$\pmb{4}$ Ve$\pmb{6}$ $\pmb{4}$.Ve$\pmb{3}$ Md$\pmb{1+}$ $\pmb{5}$.Vd$\pmb{2}$ Mb$\pmb{2}$} [$5$...Mf$2$ $6$.Ve$2$]
	\vskip 0.05cm
	\textbf{\color{cackithi}$\pmb{6}$.Xb$\pmb{4}$}
	\vskip 0.05cm
	\textbf{\color{cackithi}Bài $\pmb{3}$.
				$\pmb{1}$.Xd$\pmb{4}$!! Mb$\pmb{6}$} [$1$...Mb$2$ $2$.Ve$3$ Vf$5$ $3$.Vd$2$ Ve$5$ $4$.Xb$4$]
	\vskip 0.05cm
	\textbf{\color{cackithi}$\pmb{2}$.Ve$\pmb{5}$ Mc$\pmb{8}$ $\pmb{3}$.Ve$\pmb{6}$ Ma$\pmb{7}$ $\pmb{4}$.Vd$\pmb{7}$ Mb$\pmb{5}$} [$4$...Vf$5$ $5$.Xa$4$ Mb$5$ $6$.Xa$5$; $4$...Vg$6$ $5$.Xd$5$]
	\vskip 0.05cm
	\textbf{\color{cackithi}$\pmb{5}$.Xd$\pmb{5+}$}
	\vskip 0.05cm
	$\pmb{1-0}$
\end{multicols}
%\newpage
%\begingroup
%\AddToShipoutPicture*{\put(148,700){\includegraphics[scale=1]{../tieude1.pdf}}} 
%\centering
%\endgroup
%\vspace*{5pt}
%
%\begin{multicols}{2}
%	\setlength{\abovedisplayskip}{4pt}
%	\setlength{\belowdisplayskip}{4pt}
%	Trong phần đầu chuyên mục, chúng tôi sẽ trình bày lời giải của các bài toán trong cuộc thi Olympic toán Maclaurin năm $2021$ tại Vương quốc Anh đăng trong số báo $4/2022$. 
%	\begin{figure}[H]
%		\vspace*{-5pt}
%		\centering
%		\captionsetup{labelformat= empty, justification=centering}
%		\includegraphics[width= 0.85\linewidth]{gocolympic}
%		\vspace*{-5pt}
%	\end{figure}
%	{\bf\color{cackithi} OC$\pmb{7.}$} Cho tam giác $ABC$ như trong hình vẽ. Một đường tròn, đi qua điểm $C$ và tiếp xúc với $AB$ tại $B,$ cắt đường thẳng $AC$ tại $P$. Đường tròn thứ hai, đi qua điểm $B$ và tiếp xúc với $AC$ tại $C,$ cắt đường thẳng $AB$ tại $Q.$  Chứng minh rằng 
%	\begin{align*}
%		\frac{AP}{AQ}= \left(\frac{AB}{AC} \right)^3.
%	\end{align*}
%	\begin{figure}[H]
%		\vspace*{-5pt}
%		\centering
%		\includegraphics[width=0.8\linewidth]{OC7}
%		\vspace*{-5pt}
%	\end{figure}
%	\textit{Lời giải.} Ta nối $PB$ và $QC$ và khai thác tính chất: góc tạo bởi tiếp tuyến và dây cung bằng  góc nội tiếp chắn cung đó. Ta nhận được $ \angle BPC = \angle ABC = \angle QCA.$
%	\vskip 0.1cm
%	Như vậy các tam giác $ACQ, ABC$ and $APB$ đồng dạng vì có hai góc bằng nhau. Do đó
%	\begin{align*}
%		\frac{AQ}{AC}=\frac{AC}{AB}=\frac{AB}{AP}.
%	\end{align*}
%	Từ đây, ta nhận được $AP=\frac{AB^2}{AC}$
%	và $AQ=\frac{AC^2}{AB}$ và suy ra đẳng thức cần chứng minh. 
%	\vskip 0.1cm
%	{\bf\color{cackithi} OC$\pmb{8.}$} Một dãy số nguyên $a_1, a_2, a_3, \cdots$ được xác định bởi: 
%	\begin{align*}
%		&a_1 = k, a_{n + 1} = a_n +8n \\
%		 &\text{với mọi số nguyên}\ n \ge 1.
%	\end{align*}
%	Tìm tất cả các giá trị của $k$ sao cho mọi số hạng trong dãy đều là số chính phương.
%	\vskip 0.1cm
%	\textit{Lời giải.} 
%	Từ giả thiết ta có $a_1=k$ và $a_2=k+8$ là các số chính phương, giả sử $k=m^2$. Do chênh lệch giữa 2 số chính phương $(m+3)^2$ và $m^2$ là $(m+3)^2-m^2= m^2+6m+9-m^2=6m+9>8$ nên $k+8$ chỉ có thể bằng $(m+1)^2$ hoặc $(m+2)^2$. Tuy nhiên, nếu $k+8=(m+1)^2$ thì vô lý vì khi đó $8=(m+1)^2-m^2=2m+1$ là số lẻ. 
%	\vskip 0.1cm
%	Như vậy, chỉ còn lại khả năng duy nhất là $k+8=(m+2)^2.$ Ta suy ra $8=(m+2)^2-m^2=4m+4,$ tức là $m=1$, hay $k=1.$
%	\vskip 0.1cm
%	Ta kiểm tra lại rằng $k=1$ thỏa mãn đầu bài. Thật vậy, khi đó, ta dễ dàng tính được
%	$a_n=1+ 8(1+2\cdots (n-1))= 1+ 4n(n-1)=(2n-1)^2,$ luôn là số chính phương.   
%	\vskip 0.1cm
%	Như vậy $k=1$ là giá trị duy nhất thỏa mãn đầu bài.
%	\vskip 0.1cm
%	{\bf\color{cackithi} OC$\pmb{9.}$} Một con mèo và một con chuột lần lượt ở tại các ô trên cùng bên phải và dưới cùng bên trái của một bảng ô vuông kích thước $m\times n$, trong đó $m, n > 1.$ Mỗi giây cả hai đều di chuyển
%	chéo một ô (sang một trong các ô có chung đúng $1$ đỉnh với ô hiện tại).
%	\vskip 0.1cm
%	Với những cặp $(m, n)$ nào thì mèo và chuột có thể đồng thời đi đến cùng một ô?
%	\begin{center}
%		\begin{tikzpicture}
%			\draw[cackithi] (0,0) grid (7,4);
%			\node[inner sep=0pt] (mouse) at (0.5,0.5)
%			{\includegraphics[width=.06\textwidth]{mouse2.png}};
%			\node[inner sep=0pt] (mouse) at (6.5,3.5)
%			{\includegraphics[width=.04\textwidth]{cat1.png}};
%		\end{tikzpicture}
%	\end{center}
%	\textit{Lời giải.} Ta tô màu các ô vuông đen, trắng như bàn cờ vua.    
%	\begin{figure}[H]
%		\vspace*{-5pt}
%		\centering
%		\captionsetup{labelformat= empty, justification=centering}
%		\begin{tikzpicture}
%			\draw (0,0) grid (7,4);
%			\foreach \x in {0,2,4,6}{
%				 \foreach \y in {0,2}{
%					\draw[fill=black] (\x,\y) rectangle (\x+1, \y+1);
%				}
%			}
%			\foreach \x in {1,3,5}{
%				\foreach \y in {1,3}{
%					\draw[fill=black] (\x,\y) rectangle (\x+1, \y+1);
%				}
%			}
%		\end{tikzpicture}
%		\vspace*{-5pt}
%	\end{figure}
%	Khi $m+n$ lẻ, ô xuất phát của mèo và chuột có màu khác nhau.  Do đó, tại bất kỳ thời điểm nào, mèo và chuột luôn ở những ô khác màu và chúng không thể đến cùng một ô.  
%	\vskip 0.1cm
%	Khi $m+n$ chẵn, Ta tô màu xanh, đỏ luân phiên những ô trên đường đi của mèo và chuột như trên hình vẽ. 
%	\vskip 0.1cm
%	Trước tiên ta xét trường hợp cả $m$ và $n$ đều chẵn. 
%	\begin{figure}[H]
%		\vspace*{-5pt}
%		\centering
%		\captionsetup{labelformat= empty, justification=centering}
%		\begin{tikzpicture}
%			\draw[cackithi] (0,0) grid (6,4);
%			\foreach \x in {0,2,4}{
%				\foreach \y in {0,2}{
%					\draw[cackithi,fill=red] (\x,\y) rectangle (\x+1, \y+1);
%				}
%			}
%			\foreach \x in {1,3,5}{
%				\foreach \y in {1,3}{
%					\draw[cackithi,fill=cackithi] (\x,\y) rectangle (\x+1, \y+1);
%				}
%			}
%		\end{tikzpicture}
%		\vspace*{-5pt}
%	\end{figure}
%	Như vậy, ban đầu chuột ở ô đỏ, mèo ở ô xanh và mỗi bước chúng đều di chuyển đến những ô khác màu với ô hiện tại. Do đó mèo và chuột luôn ở những ô khác màu và không bao giờ có thể đến cùng một ô.
%	\vskip 0.1cm
%	Còn lại trường hợp cả $m$ và $n$ đều lẻ. Có cách như sau để mèo và chuột đến được cùng một ô: chuột chỉ đi tới, lui giữa ô đỏ (đánh số $1$) và ô xanh (đánh số $2$) ở góc như trong hình bên dưới, còn mèo tiến dần về phía chuột. 
%	\vskip 0.1cm
%	Chú ý rằng, ban đầu mèo và chuột xuất phát từ các ô cùng màu nên tại mọi thời điểm chúng luôn ở những ô  cùng màu. Do đó khi mèo đi đến ô xanh số $2$ thì chuột cũng phải ở một ô xanh, tức là chuột cũng ở chính ô này. 
%	\begin{figure}[H]
%		\vspace*{5pt}
%		\centering
%		\captionsetup{labelformat= empty, justification=centering}
%		\begin{tikzpicture}
%			\draw[cackithi] (0,0) grid (7,5);
%			\foreach \x in {0,2,4,6}{
%				\foreach \y in {0,2,4}{
%					\draw[cackithi,fill=red] (\x,\y) rectangle (\x+1, \y+1);
%				}
%			}
%			\foreach \x in {1,3,5}{
%				\foreach \y in {1,3}{
%					\draw[cackithi,fill=cackithi] (\x,\y) rectangle (\x+1, \y+1);
%				}
%			}
%			\draw[black] (0.5,0.5) node{$\color{black}{1}$};
%			\draw (1.5,1.5) node{$\color{black}{2}$};
%		\end{tikzpicture}
%		\vspace*{-5pt}
%	\end{figure}
%	Như vậy mèo và chuột có thể đến được cùng một ô khi và chỉ khi cả $m$ và $n$ đều lẻ. 
%	\vskip 0.1cm
%	Trong phần cuối của chuyên mục kỳ này, chúng tôi sẽ giới thiệu với bạn đọc ba bài toán trong kỳ thi Olympic Toán học trẻ của Ba Lan năm $2022$. Các bài toán này phù hợp với trình độ học sinh khối lớp $6-8$.
%	\vskip 0.1cm
%	{\bf\color{cackithi} OC$\pmb{16.}$}  Trong lớp của Marek có $17$ học sinh và tất cả đều làm một bài kiểm tra. Biết rằng điểm của Marek cao hơn $17$ điểm so với điểm trung bình của các học sinh còn lại trong lớp. Hỏi điểm của Marek cao hơn điểm trung bình của cả lớp là bao nhiêu?
%	\vskip 0.1cm
%	{\bf\color{cackithi} OC$\pmb{17.}$} Giả sử mỗi ô vuông trong bảng dưới đây được điền một số nguyên dương từ $1$ đến $17$ sao cho:
%	\vskip 0.1cm
%	-- Các số được điền đôi một phân biệt;
%	\vskip 0.1cm
%	-- Tổng của các số trong mỗi cột đều bằng nhau và tổng các số ở hàng trên cùng gấp đôi tổng các số ở hàng dưới cùng.
%	\vskip 0.1cm
%	Hỏi trong các số từ $1$ đến $17$, số nào không được điền vào bảng? Vì sao?
%	\begin{figure}[H]
%		\vspace*{-5pt}
%		\centering
%		\begin{tikzpicture}[scale=0.87]
%			\draw[cackithi] (0,0) grid (8,2);
%		\end{tikzpicture}
%		\vspace*{-5pt}
%	\end{figure}
%	{\bf\color{cackithi} OC$\pmb{18.}$} Các điểm $K, L, M$ lần lượt nằm trên các cạnh $BC, CA, AB$ của tam giác đều
%	$ABC$ và thỏa mãn các điều kiện sau:
%	\begin{align*}
%		KM = LM, \angle KML = 90^\circ \ \text{\color{black}và} \ AM = BK.
%	\end{align*}
%	Chứng minh rằng $\angle CKL = 90^\circ.$
%\end{multicols}