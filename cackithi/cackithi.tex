 \thispagestyle{cackithitoannone}
\pagestyle{cackithitoan}
\everymath{\color{cackithi}}
\graphicspath{{../cackithi/pic/}}
%\blfootnote{{\color[named]{cackithi}$^1$Trường THPT chuyên Khoa học Tự nhiên, Đại học KHTN, Đại học Quốc Gia Hà Nội.}}
\begingroup
\AddToShipoutPicture*{\put(0,616){\includegraphics[width=19.3cm]{../bannercackithi}}} 
\AddToShipoutPicture*{\put(120,525){\includegraphics[scale=1]{../tieude.pdf}}} 
\centering
\endgroup
\vspace*{190pt}

\begin{multicols}{2}
	\textbf{\color{cackithi}Vài nét về kỳ thi}
	\vskip 0.1cm
	Đây là kỳ thi được tổ chức lần đầu tiên vào năm $2000$, do bộ Giáo dục và Đào tạo Pháp và hiệp hội Animath khởi xướng, nhằm mục đích phát triển sự tò mò, khả năng tìm tòi cũng như tư duy phản biện của học sinh thông qua việc xử lý hoặc cá nhân hoặc theo nhóm một số bài toán cụ thể. Bên cạnh đó, kỳ thi còn nhằm mục đích phát triển và nâng cao văn hóa khoa học và công nghệ, nhằm mục đích kích thích sự sáng tạo và chủ động của học sinh, khơi dậy trong học sinh niềm yêu thích với bộ môn Toán. Đối tượng mà kỳ thi hướng tới là những học sinh khối lớp 11 theo chương trình phổ thông Pháp, ở cả trong và ngoài nước. 
	\vskip 0.1cm
	Học sinh sẽ trải qua hai bài thi độc lập, mỗi bài thi kéo dài $120$ phút. Tùy theo chương trình mà học sinh theo học, chương trình chuyên biệt hay chương trình phổ quát, học sinh chọn hai trong ba bài tập được đưa ra. Bài thi thứ nhất là phần thi cá nhân bao gồm những bài tập thuộc cấp quốc gia. Bài thi số hai gồm các bài tập thuộc cấp tỉnh, điểm đặc biệt của bài thi này là học sinh có thể chọn hoặc thi đơn hoặc thi theo nhóm từ $2$ tới $3$ thí sinh thậm chí các thành viên trong nhóm có thể tới từ những lớp khác nhau trong cùng một khối, đây là một trong những điểm đổi mới của kỳ thi Olympic Toán so với những kỳ thi Toán khác nhằm mục đích khuyến khích và phát huy khả năng làm việc nhóm của học sinh.
	\vskip 0.1cm 
	Hội đồng chấm thi được chỉ định và chia thành hai cấp: quốc gia và cấp tỉnh. Trong đó hội đồng cấp tỉnh một mặt chấm bài thi cấp tỉnh của các thí sinh ở tỉnh đó nhằm chọn ra những bài làm xuất sắc nhất để trao huy chương, bên cạnh đó còn chọn lựa chọn từ những bài thi cấp quốc gia của tỉnh mình những bài làm xuất sắc nhất để gửi lên hội đồng cấp quốc gia. Qua đó, hội đồng cấp quốc gia xem xét và chọn ra trong những đề cử từ các tỉnh khác nhau những bài làm xuất sắc nhất để trao huy chương. 
	\vskip 0.1cm
	Kỳ thi Olympic Toán lần thứ $22$ tại Pháp, diễn ra vào ngày $14/03/2022$ đã thu hút sự tham gia của $22000$ học sinh khối $11$ từ các trường trung học trên các tỉnh thành của khắp nước Pháp, cũng như từ những trường trung học Pháp ở nước ngoài. Sau khi thảo luận thống nhất vào ngày $25/05/2022$, hội đồng cấp quốc gia đã chọn ra $50$ bài làm xuất sắc nhất để trao giải nhất, nhì, ba cùng các huy chương và giấy chứng nhận của Bộ Giáo dục và Đào tạo Pháp. Dưới sự tài trợ của hiệp hội Animath, Viện nghiên cứu Tin học và Tự động  INRIA, Trường Bách Khoa Paris, các tập đoàn Texas instrument, Casio, Credit mutuel, cũng như từ các trường trung học có thí sinh tham gia, những học sinh đạt giải cấp quốc gia cùng gia đình và giáo viên Toán đã được mời tham dự lễ tổng kết trao giải đã diễn ra vào ngày $08/06/2022$ tại Viện nghiên cứu về thế giới Ả rập, Paris. Qua đó, học sinh có cơ hội để tìm hiểu về lịch sử của kỳ thi Olympic Toán, gặp gỡ những nhà toán học hàng đầu nước Pháp, nghe báo cáo của các giáo sư và nghiên cứu sinh đang nghiên cứu về Toán \ldots 
	\vskip 0.1cm
	\textbf{\color{cackithi}Đề thi Olympic Toán quốc gia năm $\pmb{2022}$ dành cho khu vực Pháp -- Châu Âu -- Châu Phi -- Đông Ấn} 
	\vskip 0.1cm
	Để đáp ứng các tiêu chí đã nêu của kỳ thi Olympic Toán, các đề thi được chọn lựa kỹ càng sao cho mỗi học sinh đều cảm thấy hứng thú trong việc tìm kiếm lời giải cho những bài toán mở cũng như góp phần vào việc phát huy khả năng tìm tòi, sáng tạo của từng học sinh. Vì việc lệch múi giờ giữa các địa điểm thi nên ban ra đề thi đã soạn ra ba đề thi khác nhau, tương ứng với các khu vực địa lý khác nhau, bao gồm khu vực Pháp -- Châu Âu -- Châu Phi -- Đông Ấn, khu vực  Châu Mỹ -- Antilles -- Guyana và khu vực  Châu Á -- Thái Bình Dương. Trong khuôn khổ của bài viết, các tác giả chỉ giới thiệu tới bạn đọc đề thi dành cho khu vực Pháp -- Châu Âu -- Châu Phi -- Đông Ấn, độc giả có nhu cầu tìm hiểu thêm có thể tham khảo tại đường dẫn dưới đây:
	\vskip 0.1cm
	https://www.freemaths.fr/annales-olympiad\\es-mathematiques-premieres-scientifiques-s\\/mathematiques-olympiades-epreuve-nation\\ale-enonce-2022.pdf
	\vskip 0.1cm
	\textbf{\color{cackithi}Bài $\pmb{1}$ (Chung cho tất cả thí sinh)}
	\vskip 0.1cm
	\begin{center}
		\textbf{\color{cackithi}Dán nhãn duyên dáng của một hình}
	\end{center}
	Xét một tập hợp hữu hạn các điểm. Ta nối một số điểm trong số những điểm đã cho bởi những đoạn thẳng. Tập hợp được tạo ra theo cách đó được gọi là \textit{hình}.
	\vskip 0.1cm 
	Ta thực hiện việc \textit{dán nhãn} của một hình gồm $n$ đoạn thẳng bằng cách gắn mỗi đỉnh của hình đó với một số tự nhiên đôi một khác nhau trong khoảng từ $0$ tới $n$.
	\vskip 0.1cm 
	Mỗi đoạn thẳng được gán với giá trị tuyệt đối của hiệu giữa hai số tự nhiên được gắn cho hai đầu mút của đoạn thẳng đó. Giá trị tuyệt đối thu được là một số tự nhiên, được gọi là \textit{trọng số} của đoạn thẳng. 
	\vskip 0.1cm
	Ta nói rằng sự dán nhãn của một hình là \textit{duyên dáng} nếu $n$ trọng số nhận được trên các đoạn thẳng là những số tự nhiên từ $1$ tới $n$.
	\vskip 0.1cm 
	Dưới đây là một ví dụ về sự dán nhãn duyên dáng cho một hình gồm $6$ điểm và $7$ đoạn thẳng.	
	\begin{figure}[H]
		\vspace*{-5pt}
		\centering
		\captionsetup{labelformat= empty, justification=centering}
		\begin{tikzpicture}
				\draw [cackithi,line width=0.8pt] (0.,2.)-- (4.,2.);
				\draw [cackithi,line width=0.8pt] (4.,2.)-- (4.,0.);
				\draw [cackithi,line width=0.8pt] (4.,0.)-- (0.,0.);
				\draw [cackithi,line width=0.8pt] (0.,0.)-- (0.,2.);
				\draw [cackithi,line width=0.8pt] (2.,2.)-- (2.,0.);
	
				\draw [fill=cackithi] (0.,2.) circle (2.5pt);
				\draw[color=cackithi] (-0.22,2.45) node {$0$};
				\draw [fill=cackithi] (4.,2.) circle (2.5pt);
				\draw[color=cackithi] (4.24,2.45) node {$2$};
				\draw [fill=cackithi] (4.,0.) circle (2.5pt);
				\draw[color=cackithi] (4.22,-0.4) node {$6$};
				\draw [fill=cackithi] (0.,0.) circle (2.5pt);
				\draw[color=cackithi] (-0.2,-0.4) node {$7$};
				\draw [fill=cackithi] (2.,2.) circle (2.5pt);
				\draw[color=cackithi] (2.,2.45) node {$3$};
				\draw [fill=cackithi] (2.,0.) circle (2.5pt);
				\draw[color=cackithi] (1.98,-0.4) node {$1$};
		\end{tikzpicture}
		\caption{\small\textit{\color{cackithi}Hình được dán nhãn.}}
		\vspace*{-10pt}
	\end{figure}
	\begin{figure}[H]
		\vspace*{-5pt}
		\centering
		\captionsetup{labelformat= empty, justification=centering}
		\begin{tikzpicture}
			\draw [cackithi,line width=0.8pt] (0.,2.)-- (4.,2.);
			\draw [cackithi,line width=0.8pt] (4.,2.)-- (4.,0.);
			\draw [cackithi,line width=0.8pt] (4.,0.)-- (0.,0.);
			\draw [cackithi,line width=0.8pt] (0.,0.)-- (0.,2.);
			\draw [cackithi,line width=0.8pt] (2.,2.)-- (2.,0.);
			
			\draw [fill=cackithi] (0.,2.) circle (2.5pt);
			\draw[color=cackithi] (-0.22,2.45) node {$0$};
			\draw(1,2.5) node[squarednode]{$3$};
			\draw [fill=cackithi] (4.,2.) circle (2.5pt);
			\draw[color=cackithi] (4.24,2.45) node {$2$};
			\draw(3,2.5) node[squarednode]{$1$};
			\draw [fill=cackithi] (4.,0.) circle (2.5pt);
			\draw[color=cackithi] (4.22,-0.4) node {$6$};
			\draw(3,-0.5) node[squarednode]{$5$};
			\draw(1,-0.5) node[squarednode]{$6$};
			\draw(-0.5,1) node[squarednode]{$7$};
			\draw(1.5,1) node[squarednode]{$2$};
			\draw(4.5,1) node[squarednode]{$4$};
			\draw [fill=cackithi] (0.,0.) circle (2.5pt);
			\draw[color=cackithi] (-0.2,-0.4) node {$7$};
			\draw [fill=cackithi] (2.,2.) circle (2.5pt);
			\draw[color=cackithi] (2.,2.45) node {$3$};
			\draw [fill=cackithi] (2.,0.) circle (2.5pt);
			\draw[color=cackithi] (1.98,-0.4) node {$1$};
		\end{tikzpicture}
		\caption{\small\textit{\color{cackithi}Hình được dán nhãn và trọng số.}}
		\vspace*{-10pt}
	\end{figure}
	\textbf{\color{cackithi}A. Một vài ví dụ}
	\vskip 0.1cm
	$1.$ Trong các hình dưới đây, hình nào cho ta một dán nhãn duyên dáng ?
	\begin{figure}[H]
		\vspace*{-10pt}
		\centering
		\captionsetup{labelformat= empty, justification=centering}
		\begin{tikzpicture}
			\draw [cackithi,line width=0.8pt] (0.,2.)-- (4.,2.);
			\draw [cackithi,line width=0.8pt] (4.,2.)-- (4.,0.);
			\draw [cackithi,line width=0.8pt] (4.,0.)-- (0.,0.);
			\draw [cackithi,line width=0.8pt] (0.,0.)-- (0.,2.);
			\draw [cackithi,line width=0.8pt] (2.,2.)-- (2.,0.);
			
			\draw [fill=cackithi] (0.,2.) circle (2.5pt);
			\draw[color=cackithi] (-0.22,2.45) node {$0$};
			\draw [fill=cackithi] (4.,2.) circle (2.5pt);
			\draw[color=cackithi] (4.24,2.45) node {$3$};
			\draw [fill=cackithi] (4.,0.) circle (2.5pt);
			\draw[color=cackithi] (4.22,-0.4) node {$5$};
			\draw [fill=cackithi] (0.,0.) circle (2.5pt);
			\draw[color=cackithi] (-0.2,-0.4) node {$7$};
			\draw [fill=cackithi] (2.,2.) circle (2.5pt);
			\draw[color=cackithi] (2.,2.45) node {$6$};
			\draw [fill=cackithi] (2.,0.) circle (2.5pt);
			\draw[color=cackithi] (1.98,-0.4) node {$1$};
		\end{tikzpicture}
		\vspace*{-5pt}
	\end{figure}
	\begin{figure}[H]
%		\vspace*{-5pt}
		\centering
		\captionsetup{labelformat= empty, justification=centering}
		\begin{tikzpicture}
			\draw [cackithi, line width=0.8pt] (0.,2.)-- (4.,2.);
			\draw [cackithi,line width=0.8pt] (4.,2.)-- (4.,0.);
			\draw [cackithi,line width=0.8pt] (4.,0.)-- (0.,0.);
			\draw [cackithi,line width=0.8pt] (0.,0.)-- (0.,2.);
			\draw [cackithi,line width=0.8pt] (2.,2.)-- (2.,0.);
			
			\draw [fill=cackithi] (0.,2.) circle (2.5pt);
			\draw[color=cackithi] (-0.22,2.45) node {$0$};
			\draw [fill=cackithi] (4.,2.) circle (2.5pt);
			\draw[color=cackithi] (4.24,2.45) node {$2$};
			\draw [fill=cackithi] (4.,0.) circle (2.5pt);
			\draw[color=cackithi] (4.22,-0.4) node {$4$};
			\draw [fill=cackithi] (0.,0.) circle (2.5pt);
			\draw[color=cackithi] (-0.2,-0.4) node {$7$};
			\draw [fill=cackithi] (2.,2.) circle (2.5pt);
			\draw[color=cackithi] (2.,2.45) node {$7$};
			\draw [fill=cackithi] (2.,0.) circle (2.5pt);
			\draw[color=cackithi] (1.98,-0.4) node {$2$};
		\end{tikzpicture}
		\vspace*{-10pt}
	\end{figure}
	$2.$ Bổ sung hình sau để được một dán nhãn duyên dáng.
	\begin{figure}[H]
		\vspace*{-10pt}
		\centering
		\captionsetup{labelformat= empty, justification=centering}
		\begin{tikzpicture}
				\draw [cackithi,line width=0.8pt] (-1.6180339887498947,2.9021130325903073)-- (0.,2.38);
				\draw [cackithi,line width=0.8pt] (0.,2.38)-- (-1.,1.);
				\draw [cackithi,line width=0.8pt] (0.,2.38)-- (1.,1.);
				\draw [cackithi,line width=0.8pt] (0.,2.38)-- (1.618033988749895,2.9021130325903064);
				\draw [cackithi,line width=0.8pt] (0.,4.077683537175253)-- (0.,2.38);
				\draw [cackithi,line width=0.8pt] (0.,4.077683537175253)-- (-1.6180339887498947,2.9021130325903073);
				\draw [cackithi,line width=0.8pt] (-1.6180339887498947,2.9021130325903073)-- (-1.,1.);
				\draw [cackithi,line width=0.8pt] (-1.,1.)-- (1.,1.);
				\draw [cackithi,line width=0.8pt] (1.,1.)-- (1.618033988749895,2.9021130325903064);
				\draw [cackithi,line width=0.8pt] (1.618033988749895,2.9021130325903064)-- (0.,4.077683537175253);
					\draw [fill=cackithi] (-1.,1.) circle (2.5pt);
%					\draw[color=cackithi] (-1.18624795654696,0.7809397247271022) node {$A$};
					\draw [fill=cackithi] (1.,1.) circle (2.5pt);
%					\draw[color=cackithi] (1.209997363286399,0.7471897906449423) node {$B$};
					\draw [fill=cackithi] (1.618033988749895,2.9021130325903064) circle (2.5pt);
					\draw[color=cackithi] (1.8681210778885187,3.1434351104783014) node {$9$};
					\draw [fill=cackithi] (0.,4.077683537175253) circle (2.5pt);
					\draw[color=cackithi] (0.028749670410799504,4.4428075726414615) node {$4$};
					\draw [fill=cackithi] (-1.6180339887498947,2.9021130325903073) circle (2.5pt);
					\draw[color=cackithi] (-1.9624964404366394,3.0928102093550613) node {$10$};
					\draw [fill=cackithi] (0.,2.38) circle (2.5pt);
					\draw[color=cackithi] (-0.005000263671360479,2.012812318725942) node {$0$};
		\end{tikzpicture}
		\vspace*{-10pt}
	\end{figure}
	\textbf{\color{cackithi}B. Trường hợp thẳng hàng}
	\vskip 0.1cm
	Với mỗi số tự nhiên dương $n$, ta xét hình $L_n$ gồm $n+1$ điểm thẳng hàng và $n$ đoạn thẳng được tạo thành từ các điểm kề nhau.
	\vskip 0.1cm
	Ta đề xuất sự gán nhãn duyên dáng của những điểm của hình $L_4$ như sau :
	\begin{figure}[H]
		\vspace*{-5pt}
		\centering
		\captionsetup{labelformat= empty, justification=centering}
		\begin{tikzpicture}[scale=0.85]
			\draw[cackithi, line width=0.8pt] (0,0) -- (8,0);
			\draw [fill=cackithi] (0,0) node[above]{$0$} circle (2.5pt);
			\draw [fill=cackithi] (2,0) node[above]{$4$} circle (2.5pt);
			\draw [fill=cackithi] (4,0) node[above]{$1$} circle (2.5pt);
			\draw [fill=cackithi] (6,0) node[above]{$3$} circle (2.5pt);
			\draw [fill=cackithi] (8,0) node[above]{$2$} circle (2.5pt);
			
			\draw(1,-0.5) node[squarednode]{$4$};
			\draw(3,-0.5) node[squarednode]{$3$};
			\draw(5,-0.5) node[squarednode]{$2$};
			\draw(7,-0.5) node[squarednode]{$1$};
		\end{tikzpicture}
		\vspace*{-10pt}
	\end{figure}
	$1.$ Chứng minh rằng tồn tại một dán nhãn duyên dáng cho mỗi hình $L_5,L_6$ và $L_7$.
	\vskip 0.1cm
	$2.$ Ta chấp nhận mà không chứng minh rằng tồn tại một dán nhãn duyên dáng đối với hình $L_{2022}$ sao cho điểm ngoài cùng bên trái được gắn số $0$. Hãy mô tả sự dán nhãn này. 
	\vskip 0.1cm
	\textbf{\color{cackithi}C. Trường hợp đa giác}
	\vskip 0.1cm
	$1.$ Chứng minh rằng mọi tam giác và tứ giác đều có thể được dán nhãn một cách duyên dáng. 
	\vskip 0.1cm
	$2.$ Dựa vào dán nhãn duyên dáng của hình đa giác $11$ cạnh dưới đây, hãy chỉ ra một cách dán nhãn duyên dáng của đa giác $12$ cạnh.
	\begin{figure}[H]
		\vspace*{-5pt}
		\centering
		\captionsetup{labelformat= empty, justification=centering}
		\begin{tikzpicture}[scale=0.7]
			\draw [line width=0.8pt] (4.,0.)-- (6.,0.);
			\draw [line width=0.8pt] (6.,0.)-- (7.682507065662361,1.0812816349111944);
			\draw [line width=0.8pt] (7.682507065662361,1.0812816349111944)-- (8.513337091666134,2.90054562562023);
			\draw [line width=0.8pt] (8.513337091666134,2.90054562562023)-- (8.228707415119565,4.880188509382095);
			\draw [line width=0.8pt] (8.228707415119565,4.880188509382095)-- (6.918985947228994,6.3916876580906115);
			\draw [line width=0.8pt] (6.918985947228994,6.3916876580906115)-- (5.,6.955152771773471);
			\draw [line width=0.8pt] (5.,6.955152771773471)-- (3.081014052771007,6.391687658090612);
			\draw [line width=0.8pt] (3.081014052771007,6.391687658090612)-- (1.7712925848804368,4.880188509382096);
			\draw [line width=0.8pt] (1.7712925848804368,4.880188509382096)-- (1.4866629083338658,2.9005456256202327);
			\draw [line width=0.8pt] (1.4866629083338658,2.9005456256202327)-- (2.317492934337639,1.0812816349111949);
			\draw [line width=0.8pt] (2.317492934337639,1.0812816349111949)-- (4.,0.);
		
			\draw [fill=cackithi] (4.,0.) circle (2.5pt);
			\draw[color=cackithi] (3.868571428571432,-0.35) node {$2$};
			\draw(5,-0.5) node[squarednode]{$7$};
			\draw [fill=cackithi] (6.,0.) circle (2.5pt);
			\draw[color=cackithi] (6.09714285714286,-0.35) node {$9$};
			\draw(7,-0.1) node[squarednode]{$5$};
			\draw [fill=cackithi] (7.682507065662361,1.0812816349111944) circle (2.5pt);
			\draw[color=cackithi] (8.001904761904765,0.9009523809523803) node {$4$};
			\draw(8.6,1.75) node[squarednode]{$4$};
			\draw [fill=cackithi] (8.513337091666134,2.90054562562023) circle (2.5pt);
			\draw[color=cackithi] (8.84,2.8057142857142856) node {$8$};
			\draw(9,4) node[squarednode]{$3$};
			\draw [fill=cackithi] (8.228707415119565,4.880188509382095) circle (2.5pt);
			\draw[color=cackithi] (8.573333333333336,5.0914285714285725) node {$5$};
			\draw(8.1,6) node[squarednode]{$2$};
			\draw [fill=cackithi] (6.918985947228994,6.3916876580906115) circle (2.5pt);
			\draw[color=cackithi] (7.144761904761908,6.786666666666668) node {$7$};
			\draw(6.1,7.4) node[squarednode]{$1$};
			\draw [fill=cackithi] (5.,6.955152771773471) circle (2.5pt);
			\draw[color=cackithi] (5.030476190476193,7.453333333333335) node {$6$};
			\draw(4,7.4) node[squarednode]{$6$};
			\draw [fill=cackithi] (3.081014052771007,6.391687658090612) circle (2.5pt);
			\draw[color=cackithi] (2.9161904761904793,6.824761904761906) node {$0$};
			\draw(1.8,6) node[squarednode]{$11$};
			\draw [fill=cackithi] (1.7712925848804368,4.880188509382096) circle (2.5pt);
			\draw[color=cackithi] (1.4495238095238128,5.1485714285714295) node {$11$};
			\draw(0.9,4) node[squarednode]{$10$};
			\draw [fill=cackithi] (1.4866629083338658,2.9005456256202327) circle (2.5pt);
			\draw[color=cackithi] (1.125714285714289,2.977142857142857) node {$1$};
			\draw(1.45,1.75) node[squarednode]{$9$};
			\draw [fill=cackithi] (2.317492934337639,1.0812816349111949) circle (2.5pt);
			\draw[color=cackithi] (1.9638095238095272,0.8628571428571423) node {$10$};
			\draw(3,-0.1) node[squarednode]{$8$};
		\end{tikzpicture}
		\vspace*{-10pt}
	\end{figure}
	$3.$ Xác định tính chẵn lẻ đối với trọng số của một đoạn thẳng khi mà các số được gán cho các đầu mút của nó: 
	\vskip 0.1cm
	$a.$ Khác nhau về tính chẵn lẻ.
	\vskip 0.1cm
	$b.$ Cùng tính chẵn lẻ. 
	\vskip 0.1cm
	$4.$ Từ đó suy ra rằng hình ngũ giác  không thể có dán nhãn duyên dáng. 
	\vskip 0.1cm
	\textbf{\color{cackithi}D. Một hình đa giác với số cạnh lớn}
	\vskip 0.1cm
	Ta ký hiệu $K_{2022}$ là hình được tạo thành từ $2022$ điểm sao cho mỗi cặp điểm bất kỳ trong chúng được nối với nhau bằng một đoạn thẳng duy nhất.
	\vskip 0.1cm
	$1.$ Chứng minh rằng $K_{2022}$ được tạo thành từ $2 043 231$ đoạn thẳng.
	\vskip 0.1cm
	$2.$ Giả sử rằng tồn tại một dán nhãn duyên dáng đối với hình $K_{2022}$.
	\vskip 0.1cm
	$a.$ Có bao nhiêu đoạn thẳng mang trọng số là số lẻ?
	\vskip 0.1cm
	$b.$ Ta ký hiệu $p$ là số điểm được dán nhãn là một số chẵn. Biểu diễn theo tham số $p$, số đoạn thẳng mà ở đó trọng số là một số lẻ.
	\vskip 0.1cm 
	$3.$ Chứng minh rằng hình $K_{2022}$ không thể có dán nhãn duyên dáng.
	\vskip 0.1cm
	\textbf{\color{cackithi}Bài $\pmb{2}$ ( Dành cho thí sinh theo chương trình chuyên)}
	\begin{center}
		\textbf{\color{cackithi}Những số phân chia được}
	\end{center}
	\textbf{\color{cackithi}Phần A}
	\vskip 0.1cm
	Ta nói rằng một số tự nhiên là \textit{phân chia được đơn nguyên} nếu như số đó lớn hơn hoặc bằng $3$ và viết được dưới dạng : $1+2+3+\cdots +p$, trong đó $p$ là một số tự nhiên lớn hơn hoặc bằng $2$. Ví dụ, $3$ và $10$ là những số tự nhiên phân chia được đơn nguyên bởi vì: $3 =1+2$  và $10=1+2+3+4$.
	\vskip 0.1cm 
	Ta nhắc lại rằng, tổng các số tự nhiên từ $1$ tới $n$ được cho bởi công thức: 
	\begin{align*}
		1+2+3+\cdots+n=\frac{n(n+1)}{2}.
	\end{align*}
	$1.$ $a.$ Chứng minh rằng $21$ và $136$ là những số tự nhiên phân chia được đơn nguyên.  
	\vskip 0.1cm
	$1.$ $b.$ Số tự nhiên $1850$ có phân chia được đơn nguyên không?
	\vskip 0.1cm
	$2.$ Xét $a$ là một số tự nhiên lớn hơn hoặc bằng $3$. Hãy xác định điều kiện cần và đủ sao cho $a$ là một số tự nhiên phân chia được đơn nguyên. 
	\vskip 0.1cm
	\textbf{\color{cackithi}Phần B}
	\vskip 0.1cm
	Ta nói rằng một số tự nhiên là phân chia được nếu nó có thể viết dưới dạng tổng của hai hoặc nhiều hơn số nguyên dương liên tiếp. Ví dụ, $24$ và $25$ là những số tự nhiên phân chia được vì $24 = 7 + 8 + 9$ và $25 = 12 + 13$. Tuy nhiên $4$ không phải là số tự nhiên phân chia được vì $1 + 2 < 4 < 1 + 2 + 3$ và $2 + 3 > 4$.
	\vskip 0.1cm
	$1.$ Chứng minh rằng $9$ và $15$ là những số tự nhiên phân chia được nhưng $16$ thì không.
	\vskip 0.1cm
	$2.$ Chứng minh rằng mọi số nguyên lẻ lớn hơn hoặc bằng $3$ là phân chia được. 
	\vskip 0.1cm
	Xét $k$ và $q$ là những số tự nhiên với $k\ge 2$.  Đặt $S=(q+1)+(q+2)+\cdots+(q+k)$. Chứng minh rằng: $2S=k(k+1+2q)$.
	\vskip 0.1cm
	$4.$ Chứng minh rằng mọi lũy thừa của $2$ đều không phân chia được. 
	 \vskip 0.1cm
	$5.$ Chúng ta quan tâm đến những số nguyên dương chẵn và không phải là lũy thừa của $2$. Gọi $n$ là một số như thế. Ta chấp nhận rằng tồn tại duy nhất một cặp số tự nhiên $(r,m)$ trong đó $m$ là một số tự nhiên lẻ lớn hơn hoặc bằng $3$ và $r$ một số tự nhiên lớn hơn hoặc bằng $1$, sao cho  $n=2^r\times m$.
	\vskip 0.1cm
	$a.$ Xác định $r$ và $m$ khi $n=56$. Từ đó chỉ ra rằng $56$ là một số phân chia được và hãy viết nó dưới dạng tổng của các số tự nhiên dương liên tiếp.
	\vskip 0.1cm
	$b.$ Chứng minh rằng $44$ là phân chia được.
	\vskip 0.1cm
	$c.$ Chứng minh rằng mọi  số nguyên dương chẵn và không phải là lũy thừa của $2$ là phân chia được. 
	\vskip 0.1cm
	$6.$ Từ những kết quả trên, hãy xác định tập hợp tất cả các số tự nhiên phân chia được.
	\vskip 0.1cm
	\textbf{\color{cackithi}Phần C}
	\vskip 0.1cm
	Một số tự nhiên được gọi là phân chia được một cách duy nhất nếu số đó được biểu diễn một cách duy nhất dưới dạng tổng của hai hoặc nhiều hơn số tự nhiên dương liên tiếp.
	\vskip 0.1cm
	$1.$ Chứng minh rằng $13$ là số phân chia được một cách duy nhất. Số $25$ có phải là số phân chia được một cách duy nhất không?
	\vskip 0.1cm
	$2.$ $a.$ Xét số tự nhiên $n$ là tổng của k số tự nhiên dương liên tiếp, với $k\ ge3$. Ta có thể viết $n$ dưới dạng  $n=(q+1)+(q+2)+\cdots+(q+k)$, với $q$ là số tự nhiên. Chứng minh rằng $n$ không phải là số nguyên tố.
	\vskip 0.1cm
	$b$. Từ đó suy ra rằng mọi số nguyên tố lớn hơn hoặc bằng $3$ là phân chia được một cách duy nhất.
	\vskip 0.1cm
	\textbf{\color{cackithi}Bài $\pmb{3}$ (Dành cho các thí sinh không theo chương trình chuyên)}
	\begin{center}
		\textbf{\color{cackithi}Số ba}
	\end{center}
	Ta xây dựng một dãy số tự nhiên dựa trên quy tắc sau.
	\vskip 0.1cm
	\textbf{\color{cackithi}Quy tắc}
	\vskip 0.1cm
	Số hạng đầu tiên của dãy là $4$.
	\vskip 0.1cm
	Từ một số hạng, để có số hạng tiếp theo, ta thực hiện một trong những phép toán sau:
	\vskip 0.1cm 
	$\bullet$ Nhân số đó với $3$;
	\vskip 0.1cm
	$\bullet$ Nhân số với $3$ rồi cộng kết quả nhận được với $2$;
	\vskip 0.1cm
	$\bullet$ Nếu là số chẵn thì chia cho $2$.
	\vskip 0.1cm
	Nếu một trong các dãy được xây dựng theo cách này có số hạng nào đó bằng $N$, thì ta nói rằng $N$ là \textit{số có thể đạt được}.
	\vskip 0.1cm 
	Ví dụ, số $11$ có thể đạt được: thật vậy, ta bắt đầu từ số $4$, nhân $4$ với $3$ để được $12$, sau đó ta chia $12$ cho $2$ hai lần liên tiếp để được $3$, sau đó nhân $3$ với $3$ rồi cộng $2$ ta được kết quả là $11$.
	\vskip 0.1cm 
	$1.$ Chứng tỏ rằng tất cả các số tự nhiên từ $1$ đến $12$ đều có thể đạt được bằng quy tắc nêu trên. 
	\vskip 0.1cm
	$2.$ Chứng tỏ rằng $2022$ có thể đạt được bằng quy tắc nêu trên. 
	\vskip 0.1cm
	$3.$ Giả sử rằng tồn tại các số tự nhiên không thể đạt được bằng cách áp dụng quy tắc nêu trên. Gọi $m$ là số nhỏ nhất như vậy.
	\vskip 0.1cm
	$a.$ Chứng tỏ rằng $m$ không phải là bội của $3$.
	\vskip 0.1cm
	$b.$ Chứng tỏ rằng $m-2$ không phải là bội của $3$.
	\vskip 0.1cm
	$c.$ Chứng tỏ rằng $m-1$ cũng không phải là bội của $3$.
	\vskip 0.1cm
	$d.$ Dựa vào kết quả trên, hãy đưa ra kết luận. 
	\vskip 0.1cm
	\textbf{\color{cackithi}Tài liệu tham khảo}
	\vskip 0.1cm
	[$1$] Les Olympiades nationales de mathématiques | Ministère de l'Education Nationale et de la Jeunesse
	\vskip 0.1cm
	[$2$] https://www.freemaths.fr/annales-olym\\piades-mathematiques-premieres-scientifiqu\\es-s/nationales/2022
\end{multicols}
